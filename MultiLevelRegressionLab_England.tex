% Options for packages loaded elsewhere
\PassOptionsToPackage{unicode}{hyperref}
\PassOptionsToPackage{hyphens}{url}
\PassOptionsToPackage{dvipsnames,svgnames,x11names}{xcolor}
%
\documentclass[
]{article}
\usepackage{amsmath,amssymb}
\usepackage{lmodern}
\usepackage{iftex}
\ifPDFTeX
  \usepackage[T1]{fontenc}
  \usepackage[utf8]{inputenc}
  \usepackage{textcomp} % provide euro and other symbols
\else % if luatex or xetex
  \usepackage{unicode-math}
  \defaultfontfeatures{Scale=MatchLowercase}
  \defaultfontfeatures[\rmfamily]{Ligatures=TeX,Scale=1}
\fi
% Use upquote if available, for straight quotes in verbatim environments
\IfFileExists{upquote.sty}{\usepackage{upquote}}{}
\IfFileExists{microtype.sty}{% use microtype if available
  \usepackage[]{microtype}
  \UseMicrotypeSet[protrusion]{basicmath} % disable protrusion for tt fonts
}{}
\makeatletter
\@ifundefined{KOMAClassName}{% if non-KOMA class
  \IfFileExists{parskip.sty}{%
    \usepackage{parskip}
  }{% else
    \setlength{\parindent}{0pt}
    \setlength{\parskip}{6pt plus 2pt minus 1pt}}
}{% if KOMA class
  \KOMAoptions{parskip=half}}
\makeatother
\usepackage{xcolor}
\usepackage[margin=1in]{geometry}
\usepackage{color}
\usepackage{fancyvrb}
\newcommand{\VerbBar}{|}
\newcommand{\VERB}{\Verb[commandchars=\\\{\}]}
\DefineVerbatimEnvironment{Highlighting}{Verbatim}{commandchars=\\\{\}}
% Add ',fontsize=\small' for more characters per line
\usepackage{framed}
\definecolor{shadecolor}{RGB}{248,248,248}
\newenvironment{Shaded}{\begin{snugshade}}{\end{snugshade}}
\newcommand{\AlertTok}[1]{\textcolor[rgb]{0.94,0.16,0.16}{#1}}
\newcommand{\AnnotationTok}[1]{\textcolor[rgb]{0.56,0.35,0.01}{\textbf{\textit{#1}}}}
\newcommand{\AttributeTok}[1]{\textcolor[rgb]{0.77,0.63,0.00}{#1}}
\newcommand{\BaseNTok}[1]{\textcolor[rgb]{0.00,0.00,0.81}{#1}}
\newcommand{\BuiltInTok}[1]{#1}
\newcommand{\CharTok}[1]{\textcolor[rgb]{0.31,0.60,0.02}{#1}}
\newcommand{\CommentTok}[1]{\textcolor[rgb]{0.56,0.35,0.01}{\textit{#1}}}
\newcommand{\CommentVarTok}[1]{\textcolor[rgb]{0.56,0.35,0.01}{\textbf{\textit{#1}}}}
\newcommand{\ConstantTok}[1]{\textcolor[rgb]{0.00,0.00,0.00}{#1}}
\newcommand{\ControlFlowTok}[1]{\textcolor[rgb]{0.13,0.29,0.53}{\textbf{#1}}}
\newcommand{\DataTypeTok}[1]{\textcolor[rgb]{0.13,0.29,0.53}{#1}}
\newcommand{\DecValTok}[1]{\textcolor[rgb]{0.00,0.00,0.81}{#1}}
\newcommand{\DocumentationTok}[1]{\textcolor[rgb]{0.56,0.35,0.01}{\textbf{\textit{#1}}}}
\newcommand{\ErrorTok}[1]{\textcolor[rgb]{0.64,0.00,0.00}{\textbf{#1}}}
\newcommand{\ExtensionTok}[1]{#1}
\newcommand{\FloatTok}[1]{\textcolor[rgb]{0.00,0.00,0.81}{#1}}
\newcommand{\FunctionTok}[1]{\textcolor[rgb]{0.00,0.00,0.00}{#1}}
\newcommand{\ImportTok}[1]{#1}
\newcommand{\InformationTok}[1]{\textcolor[rgb]{0.56,0.35,0.01}{\textbf{\textit{#1}}}}
\newcommand{\KeywordTok}[1]{\textcolor[rgb]{0.13,0.29,0.53}{\textbf{#1}}}
\newcommand{\NormalTok}[1]{#1}
\newcommand{\OperatorTok}[1]{\textcolor[rgb]{0.81,0.36,0.00}{\textbf{#1}}}
\newcommand{\OtherTok}[1]{\textcolor[rgb]{0.56,0.35,0.01}{#1}}
\newcommand{\PreprocessorTok}[1]{\textcolor[rgb]{0.56,0.35,0.01}{\textit{#1}}}
\newcommand{\RegionMarkerTok}[1]{#1}
\newcommand{\SpecialCharTok}[1]{\textcolor[rgb]{0.00,0.00,0.00}{#1}}
\newcommand{\SpecialStringTok}[1]{\textcolor[rgb]{0.31,0.60,0.02}{#1}}
\newcommand{\StringTok}[1]{\textcolor[rgb]{0.31,0.60,0.02}{#1}}
\newcommand{\VariableTok}[1]{\textcolor[rgb]{0.00,0.00,0.00}{#1}}
\newcommand{\VerbatimStringTok}[1]{\textcolor[rgb]{0.31,0.60,0.02}{#1}}
\newcommand{\WarningTok}[1]{\textcolor[rgb]{0.56,0.35,0.01}{\textbf{\textit{#1}}}}
\usepackage{graphicx}
\makeatletter
\def\maxwidth{\ifdim\Gin@nat@width>\linewidth\linewidth\else\Gin@nat@width\fi}
\def\maxheight{\ifdim\Gin@nat@height>\textheight\textheight\else\Gin@nat@height\fi}
\makeatother
% Scale images if necessary, so that they will not overflow the page
% margins by default, and it is still possible to overwrite the defaults
% using explicit options in \includegraphics[width, height, ...]{}
\setkeys{Gin}{width=\maxwidth,height=\maxheight,keepaspectratio}
% Set default figure placement to htbp
\makeatletter
\def\fps@figure{htbp}
\makeatother
\setlength{\emergencystretch}{3em} % prevent overfull lines
\providecommand{\tightlist}{%
  \setlength{\itemsep}{0pt}\setlength{\parskip}{0pt}}
\setcounter{secnumdepth}{-\maxdimen} % remove section numbering
\usepackage{caption}
\captionsetup[figure]{labelformat=empty}
\usepackage{booktabs}
\usepackage{longtable}
\usepackage{array}
\usepackage{multirow}
\usepackage{wrapfig}
\usepackage{float}
\usepackage{colortbl}
\usepackage{pdflscape}
\usepackage{tabu}
\usepackage{threeparttable}
\usepackage{threeparttablex}
\usepackage[normalem]{ulem}
\usepackage{makecell}
\usepackage{xcolor}
\ifLuaTeX
  \usepackage{selnolig}  % disable illegal ligatures
\fi
\IfFileExists{bookmark.sty}{\usepackage{bookmark}}{\usepackage{hyperref}}
\IfFileExists{xurl.sty}{\usepackage{xurl}}{} % add URL line breaks if available
\urlstyle{same} % disable monospaced font for URLs
\hypersetup{
  pdftitle={ESS 575: Multi-Level Regression Lab},
  pdfauthor={Team England},
  colorlinks=true,
  linkcolor={blue},
  filecolor={Maroon},
  citecolor={Blue},
  urlcolor={Blue},
  pdfcreator={LaTeX via pandoc}}

\title{ESS 575: Multi-Level Regression Lab}
\author{Team England}
\date{25 October, 2022}

\begin{document}
\maketitle

{
\hypersetup{linkcolor=}
\setcounter{tocdepth}{3}
\tableofcontents
}
Team England:

\begin{itemize}
\tightlist
\item
  Caroline Blommel
\item
  Carolyn Coyle
\item
  Bryn Crosby
\item
  George Woolsey
\end{itemize}

\href{mailto:cblommel@mail.colostate.edu}{\nolinkurl{cblommel@mail.colostate.edu}},
\href{mailto:carolynm@mail.colostate.edu}{\nolinkurl{carolynm@mail.colostate.edu}},
\href{mailto:brcrosby@rams.colostate.edu}{\nolinkurl{brcrosby@rams.colostate.edu}},
\href{mailto:george.woolsey@colostate.edu}{\nolinkurl{george.woolsey@colostate.edu}}

\hypertarget{preliminaries}{%
\section{Preliminaries}\label{preliminaries}}

\hypertarget{motivation}{%
\subsection{Motivation}\label{motivation}}

Each section of this lab has two parts -- a model \emph{building}
exercise and a model \emph{coding} exercise. The material covered here
is important and broadly useful -- building multi-levels models is a
true workhorse for understanding ecological processes because so many
problems contain information at nested spatial scales, levels of
organization, or categories. It will be worthwhile to dig in deeply to
understand it. The big picture is to demonstrate the flexibility that
you gain as a modeler by understanding basic principles of Bayesian
analysis. To accomplish that, these exercises will reinforce the
following:

\begin{enumerate}
\def\labelenumi{\arabic{enumi}.}
\tightlist
\item
  Diagramming and writing hierarchical models
\item
  Using data to model parameters
\item
  JAGS coding
\item
  Creating index variables, a critically important and useful skill
\item
  Posterior predictive checks
\end{enumerate}

\hypertarget{introduction}{%
\subsection{Introduction}\label{introduction}}

Ecological data are often collected at multiple scales or levels of
organization in nested designs. Group is a catchall term for the upper
level in many different types of nested hierarchies. Groups could
logically be composed of populations, locations, species, treatments,
life stages, and individual studies, or really, any sensible category.
We have measurements within groups on individual organisms, plots,
species, time periods, and so on. We may also have measurements on the
groups themselves, that is, covariates that apply at the upper level of
organization or spatial scale or the category that contains the
measurements. Multilevel models represent the way that a quantity of
interest responds to the combined influence of observations taken at the
group level and within the group.

Nitrous oxide \(\textrm{N} _2 \textrm{O}\), a greenhouse gas roughly 300
times more potent than carbon dioxide in forcing atmospheric warming, is
emitted when synthetic nitrogenous fertilizers are added to soils. Qian
and colleagues (2010) conducted a Bayesian meta-analysis of
\(\textrm{N} _2 \textrm{O}\) emissions (g N \(\cdot\)
ha\textsuperscript{-1} \(\cdot\) d\textsuperscript{-1}) from
agricultural soils using data from a study conducted by Carey (2007),
who reviewed 164 relevant studies. Studies occurred at different
locations, forming a group-level hierarchy (we will use only sites that
have both nitrogen and carbon data, which reduces the number of sites to
107 in the analysis here). Soil carbon content (g \(\cdot\) organic C
\(\cdot\) g\textsuperscript{-1} soil dry matter) was measured as a
group-level covariate and is assumed to be measured without error.
Observations of \(\textrm{N} _2 \textrm{O}\) emission are also assumed
to be measured without error and were paired with measurements of
fertilizer addition (kg N\(\cdot\) ha\textsuperscript{-1} \(\cdot\)
year\textsuperscript{-1}). The effect of different types of fertilizer
was also studied.

You are going to use these data to build increasingly complex models of
\(\textrm{N} _2 \textrm{O}\) emission. The initial models will ignore
some important covariates as well as how the data are structured
hierarchically into sites. This is ok! When writing for a multi-level
model like this one, do it incrementally, starting with a separate model
for each site (the no-pool model) or a model that ignores sites entirely
(the pooled model). After getting these models to work you can add
complexity by drawing the intercept for each model from a distribution,
before pursuing further refinements. We \textbf{strongly sugggest} this
approach because it is always best to do the simple thing first: there
is less to go wrong. Also, when things do go wrong it will be clearer as
to what is causing the problem.

\hypertarget{r-libraries-needed-for-this-lab}{%
\subsection{R libraries needed for this
lab}\label{r-libraries-needed-for-this-lab}}

You need to load the following libraries. Set the seed to 10 to compare
your answers to ours.

\begin{Shaded}
\begin{Highlighting}[]
\CommentTok{\# bread{-}and{-}butter}
\FunctionTok{library}\NormalTok{(tidyverse)}
\FunctionTok{library}\NormalTok{(lubridate)}
\FunctionTok{library}\NormalTok{(viridis)}
\FunctionTok{library}\NormalTok{(scales)}
\FunctionTok{library}\NormalTok{(latex2exp)}
\CommentTok{\# visualization}
\FunctionTok{library}\NormalTok{(cowplot)}
\FunctionTok{library}\NormalTok{(kableExtra)}
\CommentTok{\# jags and bayesian}
\FunctionTok{library}\NormalTok{(actuar)}
\FunctionTok{library}\NormalTok{(rjags)}
\FunctionTok{library}\NormalTok{(ggthemes)}
\FunctionTok{library}\NormalTok{(gridExtra)}
\FunctionTok{library}\NormalTok{(MCMCvis)}
\FunctionTok{library}\NormalTok{(HDInterval)}
\FunctionTok{library}\NormalTok{(BayesNSF)}
\FunctionTok{library}\NormalTok{(reshape2)}
\CommentTok{\#set seed}
\FunctionTok{set.seed}\NormalTok{(}\DecValTok{10}\NormalTok{)}
\end{Highlighting}
\end{Shaded}

\hypertarget{pooled}{%
\section{Pooled}\label{pooled}}

\hypertarget{diagramming-and-writing-the-pooled-model}{%
\subsection{Diagramming and writing the pooled
model}\label{diagramming-and-writing-the-pooled-model}}

Let's begin by ignoring the data on soil carbon and fertilizer type. In
addition, we will ignore site, such that all observations are treated as
independent from one another. This is what's known as complete pooling -
see Gelman and Hill, (2007), or just a pooled model. You will use a
linearized power function for your deterministic model of emissions as a
function of nitrogen input:

\[
\begin{aligned}
\mu_{i} = \gamma x_{i}^{\beta}\\
\alpha = \log \bigl(\gamma \bigr)\\
\log \bigl(\mu_{i} \bigr)  = \alpha+\beta \bigl(\log(x_i) \bigr)\\
g \bigl(\alpha,\beta,\log(x_i) \bigr)  = \alpha + \beta \bigl(\log(x_i) \bigr) \\
\end{aligned}
\]

\hypertarget{question-1}{%
\subsubsection{Question 1}\label{question-1}}

Interpret the coefficients \(\alpha\), \(\beta\), and \(\gamma\) in this
model.

\textcolor{violet}{We are interested in modelling $\textrm{N} _2 \textrm{O}$ emission as a function of soil carbon content, fertilizer addition, and fertilizer type. We begin by ignoring the data on soil carbon and fertilizer type. In addition, we initially ignore site-level variations by pooling the data from different sites (i.e. a pooled model). In the model $\mu_{i} = \gamma x_{i}^{\beta}$, $\gamma$ is the baseline scale factor for the fertilizer addition rate ($x_{i}$) impact to $\textrm{N} _2 \textrm{O}$ emission. The exponent $\beta$ allows for the influence of fertilizer input on $\textrm{N} _2 \textrm{O}$ emission to vary with the rate of fertilizer input. Exponential regression models are used to model situations in which growth/change begins slowly and then accelerates rapidly without bound, or where decay begins rapidly and then slows down to get closer and closer to zero. The transformation $\alpha = \log(\gamma)$ allows for linear representation of the deterministic model.}

\hypertarget{question-2}{%
\subsubsection{Question 2}\label{question-2}}

Draw a Bayesian network for a linear regression model of
\(\textrm{N} _2 \textrm{O}\) emission (\(y_{i}\)) on fertilizer addition
(\(x_{i}\)).

\begin{figure}
\includegraphics[width=0.5\linewidth,height=0.5\textheight]{../data/DAG1} \caption{DAG}\label{fig:fig.a}
\end{figure}

\hypertarget{question-3}{%
\subsubsection{Question 3}\label{question-3}}

Write out the joint distribution for a linear regression model of
\(\textrm{N} _2 \textrm{O}\) emission (\(y_{i}\)) on fertilizer addition
(\(x_{i}\)). Start by using generic \texttt{{[}\ {]}}. Use
\(\sigma^{2}\) to represent the uncertainty in your model realizing that
you might need moment matching when you choose a specific distribution.

\[
\bigl[ \alpha,\beta,\sigma^2 \mid y_i\bigr] \propto \prod_{i=1}^{n} \bigl[ \log(y_{i}) \mid g \bigl( \alpha, \beta, \log(x_i)  \bigr), \sigma^{2}\bigr][\alpha]\bigl[ \beta\bigr]\bigl[ \sigma \bigr]
\]

\hypertarget{question-4}{%
\subsubsection{Question 4}\label{question-4}}

Finish by choosing specific distributions for likelihoods and priors.
You will use the math in the answer as a template to code your model in
the subsequent exercises. What are assuming about the distribution of
the untransformed \(\mu_i\)?

\[
\bigl[ \alpha,\beta,\sigma^2 \mid y_i\bigr] \propto \prod_{i=1}^{n} {\sf normal} \bigr( \log(y_{i}) \mid g \bigl( \alpha, \beta, \log(x_i)  \bigr), \sigma^{2}\bigr) \times {\sf normal} \bigr(\alpha \mid 0,10000\bigr)  \times {\sf normal} \bigr(\beta \mid 0,10000\bigr) \times {\sf uniform}\bigr(\sigma \mid 0, 100 \bigl)
\]

\hypertarget{question-5}{%
\subsubsection{Question 5}\label{question-5}}

What is the hypothesis represented by this model?

\textcolor{violet}{We are ignoring site-level variations by pooling the data from different sites (i.e. a pooled model). This means that we are assuming that the emissions response to nitrogen addition does not vary across sites. In this pooled model, we are allowing $\textrm{N} _2 \textrm{O}$ emission to increase exponentially with fertilizer application rate.}

\hypertarget{visualizing-the-pooled-data}{%
\subsection{Visualizing the pooled
data}\label{visualizing-the-pooled-data}}

It is always a good idea to look at the data. Examine the head of the
data frame for emissions. Note that the columns \texttt{group.index} and
\texttt{fert.index} contain indices for sites and fertilizer types. We
are going to ignore these for now since the pooled model does not take
these into account. Use the code below to plot
\(\textrm{N} _2 \textrm{O}\) emissions as a function of fertilizer input
for both the logged and unlogged data.

\begin{Shaded}
\begin{Highlighting}[]
\CommentTok{\# view the first few rows of data}
\NormalTok{BayesNSF}\SpecialCharTok{::}\NormalTok{N2OEmission }\SpecialCharTok{\%\textgreater{}\%} 
  \FunctionTok{head}\NormalTok{()}
\end{Highlighting}
\end{Shaded}

\begin{verbatim}
##   fertilizer group carbon n.input emission reps group.index fert.index
## 1          A    14    2.7     180    0.620   13          10          2
## 2          A    14    4.6     180    0.450   13          10          2
## 3          A    11    0.9     112    0.230   12           7          2
## 4          A    38    0.5     100    0.153   14          29          2
## 5          A     1    4.0     250    1.000    6           1          2
## 6          A    38    0.5     100    0.216   14          29          2
\end{verbatim}

\begin{Shaded}
\begin{Highlighting}[]
\CommentTok{\# data structure}
\NormalTok{BayesNSF}\SpecialCharTok{::}\NormalTok{N2OEmission }\SpecialCharTok{\%\textgreater{}\%} 
\NormalTok{  dplyr}\SpecialCharTok{::}\FunctionTok{glimpse}\NormalTok{()}
\end{Highlighting}
\end{Shaded}

\begin{verbatim}
## Rows: 563
## Columns: 8
## $ fertilizer  <fct> A, A, A, A, A, A, A, A, A, A, A, A, A, A, A, A, A, A, A, A~
## $ group       <int> 14, 14, 11, 38, 1, 38, 11, 11, 11, 11, 13, 13, 13, 13, 1, ~
## $ carbon      <dbl> 2.70, 4.60, 0.90, 0.50, 4.00, 0.50, 0.90, 0.90, 0.90, 0.90~
## $ n.input     <dbl> 180, 180, 112, 100, 250, 100, 112, 112, 117, 82, 112, 112,~
## $ emission    <dbl> 0.620, 0.450, 0.230, 0.153, 1.000, 0.216, 0.240, 0.890, 0.~
## $ reps        <int> 13, 13, 12, 14, 6, 14, 12, 12, 12, 12, 14, 14, 14, 14, 6, ~
## $ group.index <int> 10, 10, 7, 29, 1, 29, 7, 7, 7, 7, 9, 9, 9, 9, 1, 9, 1, 7, ~
## $ fert.index  <int> 2, 2, 2, 2, 2, 2, 2, 2, 2, 2, 2, 2, 2, 2, 2, 2, 2, 2, 2, 2~
\end{verbatim}

We are going to use \texttt{ggplot} to visualize the data in this lab.
If you are unfamiliar with this package, don't worry. We will provide
you will all the codes you need and help your get oriented. We think you
will find the plotting functions in \texttt{ggplot} very powerful and
intuitive. We start by using \texttt{ggplot} to load the data frame we
will plot data from. Then we add \texttt{geom\_point} and use the
\texttt{aes} argument (the aesthetic mappings) to define the x and y
values for the points. All \texttt{ggplot} functions require you to
define the aesthetic mappings as needed. Here, they are the same as
setting x and y in the normal plot functions. The other big difference
is that \texttt{ggplot} allows you to add successive layers to the plot
using the \texttt{+} operator. You will see later on that this offers a
lot of flexibility. We add the \texttt{geom\_line} feature and then set
the theme to \texttt{minimal}. Lastly, we use the \texttt{grid.arrange}
function to position multiple plots at once. This is similar to using
\texttt{mfrow} with \texttt{par}.

\begin{Shaded}
\begin{Highlighting}[]
\CommentTok{\# untransformed}
\NormalTok{g1 }\OtherTok{\textless{}{-}} \FunctionTok{ggplot}\NormalTok{(}\AttributeTok{data =}\NormalTok{ BayesNSF}\SpecialCharTok{::}\NormalTok{N2OEmission) }\SpecialCharTok{+}
  \FunctionTok{geom\_point}\NormalTok{(}
    \AttributeTok{mapping =} \FunctionTok{aes}\NormalTok{(}\AttributeTok{y =}\NormalTok{ emission, }\AttributeTok{x =}\NormalTok{ n.input)}
\NormalTok{    , }\AttributeTok{alpha =} \DecValTok{3}\SpecialCharTok{/}\DecValTok{10}
\NormalTok{    , }\AttributeTok{shape =} \DecValTok{21}
\NormalTok{    , }\AttributeTok{colour =} \StringTok{"black"}
\NormalTok{    , }\AttributeTok{fill =} \StringTok{"brown"}
\NormalTok{    , }\AttributeTok{size =} \DecValTok{3}
\NormalTok{  ) }\SpecialCharTok{+}
  \FunctionTok{theme\_minimal}\NormalTok{()}
\CommentTok{\# log transformed}
\NormalTok{g2 }\OtherTok{\textless{}{-}} \FunctionTok{ggplot}\NormalTok{(}\AttributeTok{data =}\NormalTok{ BayesNSF}\SpecialCharTok{::}\NormalTok{N2OEmission) }\SpecialCharTok{+}
  \FunctionTok{geom\_point}\NormalTok{(}
    \AttributeTok{mapping =} \FunctionTok{aes}\NormalTok{(}\AttributeTok{y =} \FunctionTok{log}\NormalTok{(emission), }\AttributeTok{x =} \FunctionTok{log}\NormalTok{(n.input))}
\NormalTok{    , }\AttributeTok{alpha =} \DecValTok{3}\SpecialCharTok{/}\DecValTok{10}
\NormalTok{    , }\AttributeTok{shape =} \DecValTok{21}
\NormalTok{    , }\AttributeTok{colour =} \StringTok{"black"}
\NormalTok{    , }\AttributeTok{fill =} \StringTok{"brown"}
\NormalTok{    , }\AttributeTok{size =} \DecValTok{3}
\NormalTok{  ) }\SpecialCharTok{+}
  \FunctionTok{theme\_minimal}\NormalTok{() }
\CommentTok{\# plot side by side}
\NormalTok{gridExtra}\SpecialCharTok{::}\FunctionTok{grid.arrange}\NormalTok{(g1, g2, }\AttributeTok{nrow =} \DecValTok{1}\NormalTok{)}
\end{Highlighting}
\end{Shaded}

\includegraphics{C:/Data/ESS575/ess575_MultiLevelRegressionLab/src/../MultiLevelRegressionLab_England_files/figure-latex/unnamed-chunk-3-1.pdf}

\hypertarget{fitting-the-pooled-model-with-jags}{%
\subsubsection{Fitting the pooled model with
JAGS}\label{fitting-the-pooled-model-with-jags}}

You will now write a simple, pooled model where you gloss over
differences in sites and fertilizer types and lump everything into a set
of \(x\) and \(y\) pairs using the R template provided below. It is
imperative that you study the data statement and match the variable
names in your JAGS code to the left hand side of the = in the data list.
Call the intercept \texttt{alpha}, the slope \texttt{beta} and use
\texttt{sigma} to name the standard deviation in the likelihood. Also
notice, that we center the nitrogen input covariate to speed
convergence. You could also standardize this as well.

In addition to fitting this model, we would like you to have JAGS
predict the mean logged \(\textrm{N} _2 \textrm{O}\) emissions and the
median unlogged \(\textrm{N} _2 \textrm{O}\) emissions as a function of
soil fertilizer input. (Why median? Hint: think back to the distribution
of the untransformed data above in question 3 above). To help you out we
have provided the range of \(\textrm{N} _2 \textrm{O}\) values to
predict over as the third element in the \texttt{data} list. Make sure
you understand how we chose these values.

Note that in this problem and the ones that follow we have set up the
data and the initial conditions for you. This will save time and
frustration, allowing you to concentrate on writing code for the model
but you must pay attention to the names we give in the \texttt{data} and
\texttt{inits} lists. These must agree with the variable names in your
model. Please see any of the course instructors if there is anything
that you don't understand about these lists.

\begin{Shaded}
\begin{Highlighting}[]
\NormalTok{n.input.pred }\OtherTok{\textless{}{-}} \FunctionTok{seq}\NormalTok{(}\FunctionTok{min}\NormalTok{(BayesNSF}\SpecialCharTok{::}\NormalTok{N2OEmission}\SpecialCharTok{$}\NormalTok{n.input), }\FunctionTok{max}\NormalTok{(BayesNSF}\SpecialCharTok{::}\NormalTok{N2OEmission}\SpecialCharTok{$}\NormalTok{n.input), }\DecValTok{10}\NormalTok{)}

\NormalTok{data }\OtherTok{=} \FunctionTok{list}\NormalTok{(}
  \AttributeTok{log.emission =} \FunctionTok{log}\NormalTok{(BayesNSF}\SpecialCharTok{::}\NormalTok{N2OEmission}\SpecialCharTok{$}\NormalTok{emission) }\SpecialCharTok{\%\textgreater{}\%} 
      \FunctionTok{as.double}\NormalTok{()}
\NormalTok{  , }\AttributeTok{log.n.input.centered =} \FunctionTok{log}\NormalTok{(BayesNSF}\SpecialCharTok{::}\NormalTok{N2OEmission}\SpecialCharTok{$}\NormalTok{n.input) }\SpecialCharTok{{-}} 
      \FunctionTok{mean}\NormalTok{(}\FunctionTok{log}\NormalTok{(BayesNSF}\SpecialCharTok{::}\NormalTok{N2OEmission}\SpecialCharTok{$}\NormalTok{n.input)) }\SpecialCharTok{\%\textgreater{}\%} 
        \FunctionTok{as.double}\NormalTok{()}
\NormalTok{  , }\AttributeTok{log.n.input.centered.pred =} \FunctionTok{log}\NormalTok{(n.input.pred) }\SpecialCharTok{{-}} 
      \FunctionTok{mean}\NormalTok{(}\FunctionTok{log}\NormalTok{(BayesNSF}\SpecialCharTok{::}\NormalTok{N2OEmission}\SpecialCharTok{$}\NormalTok{n.input)) }\SpecialCharTok{\%\textgreater{}\%} 
        \FunctionTok{as.double}\NormalTok{()}
\NormalTok{)}

\NormalTok{inits }\OtherTok{=} \FunctionTok{list}\NormalTok{(}
  \FunctionTok{list}\NormalTok{(}\AttributeTok{alpha =} \DecValTok{0}\NormalTok{, }\AttributeTok{beta =}\NormalTok{ .}\DecValTok{5}\NormalTok{, }\AttributeTok{sigma =} \DecValTok{50}\NormalTok{)}
\NormalTok{  , }\FunctionTok{list}\NormalTok{(}\AttributeTok{alpha =} \DecValTok{1}\NormalTok{, }\AttributeTok{beta =} \FloatTok{1.5}\NormalTok{, }\AttributeTok{sigma =} \DecValTok{10}\NormalTok{)}
\NormalTok{  , }\FunctionTok{list}\NormalTok{(}\AttributeTok{alpha =} \DecValTok{2}\NormalTok{, }\AttributeTok{beta =}\NormalTok{ .}\DecValTok{75}\NormalTok{, }\AttributeTok{sigma =} \DecValTok{20}\NormalTok{)}
\NormalTok{)}
\end{Highlighting}
\end{Shaded}

\hypertarget{question-6}{%
\subsubsection{Question 6}\label{question-6}}

Write the code for the model. Compile the model and execute the MCMC to
produce a coda object. Produce trace plots of the chains for model
parameters. Produce a summary table and caterpillar plot for the
parameters and tests for convergence including the effective sample
size.

\hypertarget{jags-model}{%
\paragraph{JAGS Model}\label{jags-model}}

\begin{Shaded}
\begin{Highlighting}[]
\DocumentationTok{\#\# JAGS Model}
\NormalTok{model\{}
  
  \CommentTok{\# priors}
\NormalTok{  alpha }\SpecialCharTok{\textasciitilde{}} \FunctionTok{dnorm}\NormalTok{(}\DecValTok{0}\NormalTok{,}\FloatTok{1E{-}6}\NormalTok{)}
\NormalTok{  beta }\SpecialCharTok{\textasciitilde{}} \FunctionTok{dnorm}\NormalTok{(}\DecValTok{0}\NormalTok{,}\FloatTok{1E{-}6}\NormalTok{)}
\NormalTok{  sigma }\SpecialCharTok{\textasciitilde{}} \FunctionTok{dunif}\NormalTok{(}\DecValTok{0}\NormalTok{,}\DecValTok{100}\NormalTok{)}
\NormalTok{  tau }\OtherTok{\textless{}{-}} \DecValTok{1}\SpecialCharTok{/}\NormalTok{sigma}\SpecialCharTok{\^{}}\DecValTok{2}

  \CommentTok{\# likelihood}
  \ControlFlowTok{for}\NormalTok{ (i }\ControlFlowTok{in} \DecValTok{1}\SpecialCharTok{:}\FunctionTok{length}\NormalTok{(log.emission)) \{}
\NormalTok{    log\_mu[i] }\OtherTok{\textless{}{-}}\NormalTok{ alpha }\SpecialCharTok{+}\NormalTok{ beta }\SpecialCharTok{*}\NormalTok{ log.n.input.centered[i]}
\NormalTok{    log.emission[i] }\SpecialCharTok{\textasciitilde{}} \FunctionTok{dnorm}\NormalTok{(log\_mu[i], tau)}
\NormalTok{  \}}

  \DocumentationTok{\#\# quantities of interest}
    \CommentTok{\# predicted emissions}
    \ControlFlowTok{for}\NormalTok{ (j }\ControlFlowTok{in} \DecValTok{1}\SpecialCharTok{:}\FunctionTok{length}\NormalTok{(log.n.input.centered.pred)) \{}
\NormalTok{      log\_mu\_pred[j] }\OtherTok{\textless{}{-}}\NormalTok{ alpha }\SpecialCharTok{+}\NormalTok{ beta }\SpecialCharTok{*}\NormalTok{ log.n.input.centered.pred[j]}
\NormalTok{      mu\_pred[j] }\OtherTok{\textless{}{-}} \FunctionTok{exp}\NormalTok{(log\_mu\_pred[j])}
\NormalTok{    \}}
\NormalTok{\}}
\end{Highlighting}
\end{Shaded}

\hypertarget{implement-jags-model}{%
\paragraph{Implement JAGS Model}\label{implement-jags-model}}

\begin{Shaded}
\begin{Highlighting}[]
\DocumentationTok{\#\#\#\#\#\#\#\#\#\#\#\#\#\#\#\#\#\#\#\#\#\#\#\#\#\#\#\#\#\#\#\#\#\#\#\#\#\#\#\#\#\#\#\#\#\#\#\#\#\#\#\#\#\#\#\#\#\#\#\#\#\#\#\#\#\#}
\CommentTok{\# insert JAGS model code into an R script}
\DocumentationTok{\#\#\#\#\#\#\#\#\#\#\#\#\#\#\#\#\#\#\#\#\#\#\#\#\#\#\#\#\#\#\#\#\#\#\#\#\#\#\#\#\#\#\#\#\#\#\#\#\#\#\#\#\#\#\#\#\#\#\#\#\#\#\#\#\#\#}
\NormalTok{\{ }\CommentTok{\# Extra bracket needed only for R markdown files {-} see answers}
  \FunctionTok{sink}\NormalTok{(}\StringTok{"NO2JAGS\_pooled.R"}\NormalTok{) }\CommentTok{\# This is the file name for the jags code}
  \FunctionTok{cat}\NormalTok{(}\StringTok{"}
\StringTok{  model\{}
\StringTok{      \# priors}
\StringTok{      alpha \textasciitilde{} dnorm(0,1E{-}6)}
\StringTok{      beta \textasciitilde{} dnorm(0,1E{-}6)}
\StringTok{      sigma \textasciitilde{} dunif(0,100)}
\StringTok{      tau \textless{}{-} 1/sigma\^{}2}
\StringTok{    }
\StringTok{      \# likelihood}
\StringTok{      for (i in 1:length(log.emission)) \{}
\StringTok{        log\_mu[i] \textless{}{-} alpha + beta * log.n.input.centered[i]}
\StringTok{        log.emission[i] \textasciitilde{} dnorm(log\_mu[i], tau)}
\StringTok{      \}}
\StringTok{    }
\StringTok{      \#\# quantities of interest}
\StringTok{        \# predicted emissions}
\StringTok{        for (j in 1:length(log.n.input.centered.pred)) \{}
\StringTok{          log\_mu\_pred[j] \textless{}{-} alpha + beta * log.n.input.centered.pred[j]}
\StringTok{          mu\_pred[j] \textless{}{-} exp(log\_mu\_pred[j])}
\StringTok{        \}}
\StringTok{  \}}
\StringTok{  "}\NormalTok{, }\AttributeTok{fill =} \ConstantTok{TRUE}\NormalTok{)}
  \FunctionTok{sink}\NormalTok{()}
\NormalTok{\}}
\DocumentationTok{\#\#\#\#\#\#\#\#\#\#\#\#\#\#\#\#\#\#\#\#\#\#\#\#\#\#\#\#\#\#\#\#\#\#\#\#\#\#\#\#\#\#\#\#\#\#\#\#\#\#\#\#\#\#\#\#\#\#\#\#\#\#\#\#}
\CommentTok{\# implement model}
\DocumentationTok{\#\#\#\#\#\#\#\#\#\#\#\#\#\#\#\#\#\#\#\#\#\#\#\#\#\#\#\#\#\#\#\#\#\#\#\#\#\#\#\#\#\#\#\#\#\#\#\#\#\#\#\#\#\#\#\#\#\#\#\#\#\#\#\#\#\#}
\CommentTok{\# specify 3 scalars, n.adapt, n.update, and n.iter}
\CommentTok{\# n.adapt = number of iterations that JAGS will use to choose the sampler }
  \CommentTok{\# and to assure optimum mixing of the MCMC chain}
\NormalTok{n.adapt }\OtherTok{=} \DecValTok{1000}
\CommentTok{\# n.update = number of iterations that will be discarded to allow the chain to }
\CommentTok{\#   converge before iterations are stored (aka, burn{-}in)}
\NormalTok{n.update }\OtherTok{=} \DecValTok{10000}
\CommentTok{\# n.iter = number of iterations that will be stored in the }
  \CommentTok{\# final chain as samples from the posterior distribution}
\NormalTok{n.iter }\OtherTok{=} \DecValTok{10000}
\DocumentationTok{\#\#\#\#\#\#\#\#\#\#\#\#\#\#\#\#\#\#\#\#\#\#}
\CommentTok{\# Call to JAGS}
\DocumentationTok{\#\#\#\#\#\#\#\#\#\#\#\#\#\#\#\#\#\#\#\#\#\#}
\NormalTok{jm }\OtherTok{=}\NormalTok{ rjags}\SpecialCharTok{::}\FunctionTok{jags.model}\NormalTok{(}
  \AttributeTok{file =} \StringTok{"NO2JAGS\_pooled.R"}
\NormalTok{  , }\AttributeTok{data =}\NormalTok{ data}
\NormalTok{  , }\AttributeTok{inits =}\NormalTok{ inits}
\NormalTok{  , }\AttributeTok{n.chains =} \FunctionTok{length}\NormalTok{(inits)}
\NormalTok{  , }\AttributeTok{n.adapt =}\NormalTok{ n.adapt}
\NormalTok{)}
\end{Highlighting}
\end{Shaded}

\begin{verbatim}
## Compiling model graph
##    Resolving undeclared variables
##    Allocating nodes
## Graph information:
##    Observed stochastic nodes: 563
##    Unobserved stochastic nodes: 3
##    Total graph size: 1854
## 
## Initializing model
\end{verbatim}

\begin{Shaded}
\begin{Highlighting}[]
\NormalTok{stats}\SpecialCharTok{::}\FunctionTok{update}\NormalTok{(jm, }\AttributeTok{n.iter =}\NormalTok{ n.update)}
\CommentTok{\# save the coda object (more precisely, an mcmc.list object) to R as "zc"}
\NormalTok{zc\_pooled }\OtherTok{=}\NormalTok{ rjags}\SpecialCharTok{::}\FunctionTok{coda.samples}\NormalTok{(}
  \AttributeTok{model =}\NormalTok{ jm}
\NormalTok{  , }\AttributeTok{variable.names =} \FunctionTok{c}\NormalTok{(}\StringTok{"alpha"}\NormalTok{, }\StringTok{"beta"}\NormalTok{, }\StringTok{"sigma"}\NormalTok{, }\StringTok{"tau"}\NormalTok{, }\StringTok{"log\_mu\_pred"}\NormalTok{, }\StringTok{"mu\_pred"}\NormalTok{)}
  \CommentTok{\# , variable.names = c("a", "b", "p")}
\NormalTok{  , }\AttributeTok{n.iter =}\NormalTok{ n.iter}
\NormalTok{  , }\AttributeTok{n.thin =} \DecValTok{1}
\NormalTok{)}
\end{Highlighting}
\end{Shaded}

\hypertarget{model-output}{%
\paragraph{Model Output}\label{model-output}}

Produce trace plots of the chains for model parameters. Produce a
summary table and caterpillar plot for the parameters and tests for
convergence including the effective sample size.

\begin{Shaded}
\begin{Highlighting}[]
\DocumentationTok{\#\#\#\#\#\#\#\#\#\#\#\#\#\#\#\#\#\#\#\#\#}
\CommentTok{\# check output}
\DocumentationTok{\#\#\#\#\#\#\#\#\#\#\#\#\#\#\#\#\#\#\#\#\#}
\CommentTok{\# trace plot}
\NormalTok{MCMCvis}\SpecialCharTok{::}\FunctionTok{MCMCtrace}\NormalTok{(zc\_pooled, }\AttributeTok{params =} \FunctionTok{c}\NormalTok{(}\StringTok{"alpha"}\NormalTok{, }\StringTok{"beta"}\NormalTok{, }\StringTok{"sigma"}\NormalTok{), }\AttributeTok{pdf =} \ConstantTok{FALSE}\NormalTok{)}
\end{Highlighting}
\end{Shaded}

\includegraphics{C:/Data/ESS575/ess575_MultiLevelRegressionLab/src/../MultiLevelRegressionLab_England_files/figure-latex/unnamed-chunk-7-1.pdf}

\begin{Shaded}
\begin{Highlighting}[]
\CommentTok{\# summary}
\NormalTok{MCMCvis}\SpecialCharTok{::}\FunctionTok{MCMCsummary}\NormalTok{(zc\_pooled, }\AttributeTok{params =} \FunctionTok{c}\NormalTok{(}\StringTok{"alpha"}\NormalTok{, }\StringTok{"beta"}\NormalTok{, }\StringTok{"sigma"}\NormalTok{))}
\end{Highlighting}
\end{Shaded}

\begin{verbatim}
##             mean         sd        2.5%        50%     97.5% Rhat n.eff
## alpha 0.04367949 0.06830640 -0.09024603 0.04398248 0.1778201    1 30896
## beta  0.91191229 0.10785096  0.70148553 0.91198217 1.1231760    1 31156
## sigma 1.62618035 0.04909531  1.53376688 1.62533710 1.7256600    1 17775
\end{verbatim}

\begin{Shaded}
\begin{Highlighting}[]
\CommentTok{\# Caterpillar plots}
\NormalTok{MCMCvis}\SpecialCharTok{::}\FunctionTok{MCMCplot}\NormalTok{(zc\_pooled, }\AttributeTok{params =} \FunctionTok{c}\NormalTok{(}\StringTok{"alpha"}\NormalTok{, }\StringTok{"beta"}\NormalTok{, }\StringTok{"sigma"}\NormalTok{))}
\end{Highlighting}
\end{Shaded}

\includegraphics{C:/Data/ESS575/ess575_MultiLevelRegressionLab/src/../MultiLevelRegressionLab_England_files/figure-latex/unnamed-chunk-7-2.pdf}

\hypertarget{visualizing-the-pooled-model-predictions}{%
\subsection{Visualizing the pooled model
predictions}\label{visualizing-the-pooled-model-predictions}}

Let's overlay the predicted mean logged \(\textrm{N} _2 \textrm{O}\)
emissions and median unlogged \(\textrm{N} _2 \textrm{O}\) emissions as
a function of soil fertilizer input from the pooled model on top of the
raw data. We summarize the predictions using \texttt{MCMCpstr()} twice -
once to get the 95\% HDPI intervals and a second time to get the
posterior median for each fertilizer input value. We combine these
predictions into two data frames, one for the logged
\(\textrm{N} _2 \textrm{O}\) emissions and one for untransformed
\(\textrm{N} _2 \textrm{O}\) emissions. We append our new graphical
elements onto our old plots with the \texttt{+} operator. We plot the
median of the posterior distribution as a black line with
\texttt{geom\_line()} and the 95\% credible intervals as a yellow shaded
region using the \texttt{geom\_ribbon()} function. These data come from
a different data frame than the one we used to plot the raw data so we
need to add the \texttt{data} argument in the new \texttt{geom\_line}
and \texttt{geom\_ribbon}. Again, we provide you with the code to do
this to save time. You will need to modify this code to make similar
plots for models you fit in later exercises.

\begin{Shaded}
\begin{Highlighting}[]
\CommentTok{\# highest posterior density interval of predictions}
\NormalTok{pred1 }\OtherTok{\textless{}{-}}\NormalTok{ MCMCvis}\SpecialCharTok{::}\FunctionTok{MCMCpstr}\NormalTok{(}
\NormalTok{  zc\_pooled}
\NormalTok{  , }\AttributeTok{params =} \FunctionTok{c}\NormalTok{(}\StringTok{"mu\_pred"}\NormalTok{, }\StringTok{"log\_mu\_pred"}\NormalTok{)}
\NormalTok{  , }\AttributeTok{func =} \ControlFlowTok{function}\NormalTok{(x) HDInterval}\SpecialCharTok{::}\FunctionTok{hdi}\NormalTok{(x, .}\DecValTok{95}\NormalTok{)}
\NormalTok{)}
\CommentTok{\# median of predictions}
\NormalTok{pred2 }\OtherTok{\textless{}{-}}\NormalTok{ MCMCvis}\SpecialCharTok{::}\FunctionTok{MCMCpstr}\NormalTok{(}
\NormalTok{  zc\_pooled}
\NormalTok{  , }\AttributeTok{params =} \FunctionTok{c}\NormalTok{(}\StringTok{"mu\_pred"}\NormalTok{, }\StringTok{"log\_mu\_pred"}\NormalTok{)}
\NormalTok{  , }\AttributeTok{func =}\NormalTok{ median}
\NormalTok{)}
\CommentTok{\# put in data frame}
\NormalTok{pred.po.df }\OtherTok{\textless{}{-}}\NormalTok{ dplyr}\SpecialCharTok{::}\FunctionTok{bind\_cols}\NormalTok{(}
\NormalTok{  n.input.pred}
\NormalTok{  , }\FunctionTok{data.frame}\NormalTok{(pred1}\SpecialCharTok{$}\NormalTok{mu\_pred)}
\NormalTok{  , }\AttributeTok{median =}\NormalTok{ pred2}\SpecialCharTok{$}\NormalTok{mu\_pred}
\NormalTok{)}
\NormalTok{lpred.po.df }\OtherTok{\textless{}{-}}\NormalTok{ dplyr}\SpecialCharTok{::}\FunctionTok{bind\_cols}\NormalTok{(}
  \AttributeTok{log.n.input.pred =} \FunctionTok{log}\NormalTok{(n.input.pred)}
\NormalTok{  , }\FunctionTok{data.frame}\NormalTok{(pred1}\SpecialCharTok{$}\NormalTok{log\_mu\_pred)}
\NormalTok{  , }\AttributeTok{median =}\NormalTok{ pred2}\SpecialCharTok{$}\NormalTok{log\_mu\_pred}
\NormalTok{)}
\end{Highlighting}
\end{Shaded}

Plot the predictions

\begin{Shaded}
\begin{Highlighting}[]
\NormalTok{g3 }\OtherTok{\textless{}{-}}\NormalTok{ g1 }\SpecialCharTok{+}
  \FunctionTok{geom\_line}\NormalTok{(}
    \AttributeTok{data =}\NormalTok{ pred.po.df}
\NormalTok{    , }\AttributeTok{mapping =} \FunctionTok{aes}\NormalTok{(}\AttributeTok{x =}\NormalTok{ n.input.pred, }\AttributeTok{y =}\NormalTok{ median)}
\NormalTok{  ) }\SpecialCharTok{+}
  \FunctionTok{geom\_ribbon}\NormalTok{(}
    \AttributeTok{data =}\NormalTok{ pred.po.df}
\NormalTok{    , }\AttributeTok{mapping =} \FunctionTok{aes}\NormalTok{(}\AttributeTok{x =}\NormalTok{ n.input.pred, }\AttributeTok{ymin =}\NormalTok{ lower, }\AttributeTok{ymax =}\NormalTok{ upper)}
\NormalTok{    , }\AttributeTok{alpha =} \FloatTok{0.2}
\NormalTok{    , }\AttributeTok{fill =} \StringTok{"yellow"}
\NormalTok{  )}

\NormalTok{g4 }\OtherTok{\textless{}{-}}\NormalTok{ g2 }\SpecialCharTok{+}
  \FunctionTok{geom\_line}\NormalTok{(}
    \AttributeTok{data =}\NormalTok{ lpred.po.df}
\NormalTok{    , }\AttributeTok{mapping =} \FunctionTok{aes}\NormalTok{(}\AttributeTok{x =}\NormalTok{ log.n.input.pred, }\AttributeTok{y =}\NormalTok{ median)}
\NormalTok{  ) }\SpecialCharTok{+}
  \FunctionTok{geom\_ribbon}\NormalTok{(}
    \AttributeTok{data =}\NormalTok{ lpred.po.df}
\NormalTok{    , }\AttributeTok{mapping =} \FunctionTok{aes}\NormalTok{(}\AttributeTok{x =}\NormalTok{ log.n.input.pred, }\AttributeTok{ymin =}\NormalTok{ lower, }\AttributeTok{ymax =}\NormalTok{ upper)}
\NormalTok{    , }\AttributeTok{alpha =} \FloatTok{0.2}
\NormalTok{    , }\AttributeTok{fill =} \StringTok{"yellow"}
\NormalTok{  )}

\NormalTok{gridExtra}\SpecialCharTok{::}\FunctionTok{grid.arrange}\NormalTok{(g3, g4, }\AttributeTok{nrow =} \DecValTok{1}\NormalTok{)}
\end{Highlighting}
\end{Shaded}

\includegraphics{C:/Data/ESS575/ess575_MultiLevelRegressionLab/src/../MultiLevelRegressionLab_England_files/figure-latex/unnamed-chunk-11-1.pdf}

\hypertarget{non-pooled}{%
\section{Non-Pooled}\label{non-pooled}}

\hypertarget{diagramming-and-writing-the-no-pool-model}{%
\subsection{Diagramming and writing the no-pool
model}\label{diagramming-and-writing-the-no-pool-model}}

Great! - you've got the pooled model fitted and made some predictions
from it. However, perhaps the idea of ignoring the site effects is not
sitting so well with you. Let's take this a step further by modeling the
relationship between \(\textrm{N} _2 \textrm{O}\) emission and
fertilizer input such that the intercept \(\alpha_{j}\) varies by site
(we will again ignore the data on soil carbon and fertilizer type). This
is the opposite of the pooled model where we completely ignored the
effect of site as here we treat the intercept for each site as
independent. This is commonly called a no-pool model. The deterministic
portion of this model remains a linearized power function, but two
subscripts are required: \(i\) which indexes the measurement within
sites and \(j\) which indexes site itself.

\[
\begin{aligned}
\mu_{ij}  = \gamma_{j} x_{ij}^{\beta}\\
\alpha_{j}  = \log \bigl(\gamma_{j} \bigr)\\
\log \bigl(\mu_{ij} \bigr)  = \alpha_{j}+\beta \bigl(\log(x_{ij}) \bigr)\\
g \bigl(\alpha_{j},\beta,\log(x_{ij}) \bigr)  = \alpha_{j}+\beta \bigl(\log(x_{ij}) \bigr) \\
\end{aligned}
\]

\hypertarget{question-1-1}{%
\subsubsection{Question 1}\label{question-1-1}}

Draw a Bayesian network for a linear regression model of
\(\textrm{N} _2 \textrm{O}\) emission (\(y_{ij}\)) on fertilizer
addition (\(x_{ij}\)).

\begin{figure}

{\centering \includegraphics[width=0.5\linewidth,height=0.5\textheight]{../data/DAG2} 

}

\caption{DAG}\label{fig:unnamed-chunk-12}
\end{figure}

\hypertarget{question-2-1}{%
\subsubsection{Question 2}\label{question-2-1}}

Write out the joint distribution for a linear regression model of
\(\textrm{N} _2 \textrm{O}\) emission (\(y_{ij}\)) on fertilizer
addition (\(x_{ij}\)). Start by using generic \texttt{{[}\ {]}}. Use
\(\sigma^{2}\) to represent the uncertainty in your model realizing that
you might need moment matching when you choose a specific distribution.

\[
\bigl[ \boldsymbol{\alpha},\beta,\sigma^2 \mid \boldsymbol{y} \bigr] \propto \prod_{i=1}^{n} \prod_{j=1}^{J}  \bigl[ \log(y_{ij})\mid g\bigl(\alpha_{j},\beta,\log(x_{ij})\bigr),\sigma^{2} \bigr] \bigl[\alpha_{j} \bigr] \bigl[ \beta \bigr] \bigl[ \sigma  \bigr]
\]

\hypertarget{question-3-1}{%
\subsubsection{Question 3}\label{question-3-1}}

Finish by choosing specific distributions for likelihoods and priors.
You will use the math in the answer as a template to code your model in
the subsequent exercises.

\[
\begin{aligned}
\bigl[ \boldsymbol{\alpha},\beta,\sigma^2 \mid \boldsymbol{y} \bigr] \propto \prod_{i=1}^{n}  \prod_{j=1}^{J} {\sf normal} \bigr( \log(y_{ij}) \mid g \bigl( \alpha_{j}, \beta, \log(x_{ij})  \bigr), \sigma^{2}\bigr)\\
\times \; {\sf normal} \bigr(\alpha_{j} \mid 0,10000\bigr) \\ 
\times \; {\sf normal} \bigr(\beta \mid 0,10000\bigr) \\
\times \; {\sf uniform}\bigr(\sigma \mid 0, 100 \bigl)
\end{aligned}
\]

\hypertarget{question-4-1}{%
\subsubsection{Question 4}\label{question-4-1}}

What is the hypothesis represented by this model?

\textcolor{violet}{fill this in!!!!}

\hypertarget{visualizing-the-data}{%
\subsection{Visualizing the data}\label{visualizing-the-data}}

Let's visualize the data again, but this time highlighting the role site
plays in determining the relationship between
\(\textrm{N} _2 \textrm{O}\) emission and fertilizer input. First,
\texttt{head()} the data to see how groups are organized. You will use
\texttt{group.index} to group the observations by site.

\begin{Shaded}
\begin{Highlighting}[]
\CommentTok{\# view the first few rows of data}
\NormalTok{BayesNSF}\SpecialCharTok{::}\NormalTok{N2OEmission }\SpecialCharTok{\%\textgreater{}\%} 
  \FunctionTok{head}\NormalTok{()}
\end{Highlighting}
\end{Shaded}

\begin{verbatim}
##   fertilizer group carbon n.input emission reps group.index fert.index
## 1          A    14    2.7     180    0.620   13          10          2
## 2          A    14    4.6     180    0.450   13          10          2
## 3          A    11    0.9     112    0.230   12           7          2
## 4          A    38    0.5     100    0.153   14          29          2
## 5          A     1    4.0     250    1.000    6           1          2
## 6          A    38    0.5     100    0.216   14          29          2
\end{verbatim}

\begin{Shaded}
\begin{Highlighting}[]
\CommentTok{\# data structure}
\NormalTok{BayesNSF}\SpecialCharTok{::}\NormalTok{N2OEmission }\SpecialCharTok{\%\textgreater{}\%} 
\NormalTok{  dplyr}\SpecialCharTok{::}\FunctionTok{glimpse}\NormalTok{()}
\end{Highlighting}
\end{Shaded}

\begin{verbatim}
## Rows: 563
## Columns: 8
## $ fertilizer  <fct> A, A, A, A, A, A, A, A, A, A, A, A, A, A, A, A, A, A, A, A~
## $ group       <int> 14, 14, 11, 38, 1, 38, 11, 11, 11, 11, 13, 13, 13, 13, 1, ~
## $ carbon      <dbl> 2.70, 4.60, 0.90, 0.50, 4.00, 0.50, 0.90, 0.90, 0.90, 0.90~
## $ n.input     <dbl> 180, 180, 112, 100, 250, 100, 112, 112, 117, 82, 112, 112,~
## $ emission    <dbl> 0.620, 0.450, 0.230, 0.153, 1.000, 0.216, 0.240, 0.890, 0.~
## $ reps        <int> 13, 13, 12, 14, 6, 14, 12, 12, 12, 12, 14, 14, 14, 14, 6, ~
## $ group.index <int> 10, 10, 7, 29, 1, 29, 7, 7, 7, 7, 9, 9, 9, 9, 1, 9, 1, 7, ~
## $ fert.index  <int> 2, 2, 2, 2, 2, 2, 2, 2, 2, 2, 2, 2, 2, 2, 2, 2, 2, 2, 2, 2~
\end{verbatim}

Use the code below to plot logged \(\textrm{N} _2 \textrm{O}\) emissions
against logged fertilizer input. This is the same ggplot code as before
except now we amend it to make plots for individual sites simply by
adding the \texttt{facet\_wrap} function and specifying the grouping
variable(here it is \texttt{group.index}) as an argument.

\begin{Shaded}
\begin{Highlighting}[]
\NormalTok{g2 }\SpecialCharTok{+} \FunctionTok{facet\_wrap}\NormalTok{(}\SpecialCharTok{\textasciitilde{}}\NormalTok{group.index)}
\end{Highlighting}
\end{Shaded}

\includegraphics{C:/Data/ESS575/ess575_MultiLevelRegressionLab/src/../MultiLevelRegressionLab_England_files/figure-latex/unnamed-chunk-14-1.pdf}

\hypertarget{fitting-the-no-pool-model-with-jags}{%
\subsubsection{Fitting the no-pool model with
JAGS}\label{fitting-the-no-pool-model-with-jags}}

You will now write a simple, no-pool model using the R template provided
below. In addition to fitting this model, we would like you to have JAGS
predict the mean logged \(\textrm{N} _2 \textrm{O}\) emissions
\textbf{for each site} as a function of soil fertilizer input. To help
you out we have provided the range of \(\textrm{N} _2 \textrm{O}\)
values to predict over as the third element in the \texttt{data} list.
\textbf{Note that you must use the index trick covered in lecture to
align observations in each site with the appropriate intercept}. Here
are the preliminaries to set up the model:

\begin{Shaded}
\begin{Highlighting}[]
\NormalTok{n.sites }\OtherTok{\textless{}{-}} \FunctionTok{length}\NormalTok{(}\FunctionTok{unique}\NormalTok{(BayesNSF}\SpecialCharTok{::}\NormalTok{N2OEmission}\SpecialCharTok{$}\NormalTok{group.index))}
\NormalTok{n.input.pred }\OtherTok{\textless{}{-}} \FunctionTok{seq}\NormalTok{(}\FunctionTok{min}\NormalTok{(BayesNSF}\SpecialCharTok{::}\NormalTok{N2OEmission}\SpecialCharTok{$}\NormalTok{n.input), }\FunctionTok{max}\NormalTok{(BayesNSF}\SpecialCharTok{::}\NormalTok{N2OEmission}\SpecialCharTok{$}\NormalTok{n.input), }\DecValTok{10}\NormalTok{)}

\NormalTok{data }\OtherTok{=} \FunctionTok{list}\NormalTok{(}
  \AttributeTok{log.emission =} \FunctionTok{log}\NormalTok{(BayesNSF}\SpecialCharTok{::}\NormalTok{N2OEmission}\SpecialCharTok{$}\NormalTok{emission) }\SpecialCharTok{\%\textgreater{}\%} \FunctionTok{as.double}\NormalTok{()}
\NormalTok{  , }\AttributeTok{log.n.input.centered =} \FunctionTok{log}\NormalTok{(BayesNSF}\SpecialCharTok{::}\NormalTok{N2OEmission}\SpecialCharTok{$}\NormalTok{n.input) }\SpecialCharTok{{-}} 
      \FunctionTok{mean}\NormalTok{(}\FunctionTok{log}\NormalTok{(BayesNSF}\SpecialCharTok{::}\NormalTok{N2OEmission}\SpecialCharTok{$}\NormalTok{n.input)) }\SpecialCharTok{\%\textgreater{}\%} 
        \FunctionTok{as.double}\NormalTok{()}
\NormalTok{  , }\AttributeTok{log.n.input.centered.pred =} \FunctionTok{log}\NormalTok{(n.input.pred) }\SpecialCharTok{{-}} 
      \FunctionTok{mean}\NormalTok{(}\FunctionTok{log}\NormalTok{(BayesNSF}\SpecialCharTok{::}\NormalTok{N2OEmission}\SpecialCharTok{$}\NormalTok{n.input)) }\SpecialCharTok{\%\textgreater{}\%} 
        \FunctionTok{as.double}\NormalTok{()}
\NormalTok{  , }\AttributeTok{group =}\NormalTok{ BayesNSF}\SpecialCharTok{::}\NormalTok{N2OEmission}\SpecialCharTok{$}\NormalTok{group.index }\SpecialCharTok{\%\textgreater{}\%} \FunctionTok{as.double}\NormalTok{()}
\NormalTok{  , }\AttributeTok{n.sites =}\NormalTok{ n.sites}
\NormalTok{)}

\NormalTok{inits }\OtherTok{=} \FunctionTok{list}\NormalTok{(}
  \FunctionTok{list}\NormalTok{(}\AttributeTok{alpha =} \FunctionTok{rep}\NormalTok{(}\DecValTok{0}\NormalTok{, n.sites), }\AttributeTok{beta =}\NormalTok{ .}\DecValTok{5}\NormalTok{, }\AttributeTok{sigma =} \DecValTok{50}\NormalTok{)}
\NormalTok{  , }\FunctionTok{list}\NormalTok{(}\AttributeTok{alpha =} \FunctionTok{rep}\NormalTok{(}\DecValTok{1}\NormalTok{, n.sites), }\AttributeTok{beta =} \FloatTok{1.5}\NormalTok{, }\AttributeTok{sigma =} \DecValTok{10}\NormalTok{)}
\NormalTok{  , }\FunctionTok{list}\NormalTok{(}\AttributeTok{alpha =} \FunctionTok{rep}\NormalTok{(}\SpecialCharTok{{-}}\DecValTok{1}\NormalTok{, n.sites), }\AttributeTok{beta =}\NormalTok{ .}\DecValTok{75}\NormalTok{, }\AttributeTok{sigma =} \DecValTok{20}\NormalTok{)}
\NormalTok{)}
\end{Highlighting}
\end{Shaded}

\hypertarget{question-5-1}{%
\subsubsection{Question 5}\label{question-5-1}}

Write the code for the model. Compile the model and execute the MCMC to
produce a coda object. Produce trace plots of the chains for model
parameters, excluding \(\boldsymbol{\alpha}\) and a summary table of
these same parameters. Assess convergence and look at the effective
sample sizes for each of these parameters. Do you think any of the
chains need to be run for longer and if so why? Make a horizontal
caterpillar plot for the the \(\boldsymbol{\alpha}\).

\hypertarget{jags-model-1}{%
\paragraph{JAGS Model}\label{jags-model-1}}

\begin{Shaded}
\begin{Highlighting}[]
\DocumentationTok{\#\# JAGS Model}
\NormalTok{model\{}
  \CommentTok{\# priors}
  \CommentTok{\# allow the intercept alpha to vary across sites}
    \ControlFlowTok{for}\NormalTok{(j }\ControlFlowTok{in} \DecValTok{1}\SpecialCharTok{:}\NormalTok{n.sites)\{}
\NormalTok{      alpha[j] }\SpecialCharTok{\textasciitilde{}} \FunctionTok{dnorm}\NormalTok{(}\DecValTok{0}\NormalTok{,}\FloatTok{1E{-}6}\NormalTok{) }
\NormalTok{    \}}
  \CommentTok{\# the slope beta is constant across sites}
\NormalTok{  beta }\SpecialCharTok{\textasciitilde{}} \FunctionTok{dnorm}\NormalTok{(}\DecValTok{0}\NormalTok{,}\FloatTok{1E{-}6}\NormalTok{)}
\NormalTok{  sigma }\SpecialCharTok{\textasciitilde{}} \FunctionTok{dunif}\NormalTok{(}\DecValTok{0}\NormalTok{,}\DecValTok{100}\NormalTok{)}
\NormalTok{  tau }\OtherTok{\textless{}{-}} \DecValTok{1}\SpecialCharTok{/}\NormalTok{sigma}\SpecialCharTok{\^{}}\DecValTok{2}

  \CommentTok{\# likelihood}
  \ControlFlowTok{for}\NormalTok{(i }\ControlFlowTok{in} \DecValTok{1}\SpecialCharTok{:}\FunctionTok{length}\NormalTok{(log.emission)) \{}
\NormalTok{    log\_mu[i] }\OtherTok{\textless{}{-}}\NormalTok{ alpha[group[i]] }\SpecialCharTok{+}\NormalTok{ beta }\SpecialCharTok{*}\NormalTok{ log.n.input.centered[i]}
\NormalTok{    log.emission[i] }\SpecialCharTok{\textasciitilde{}} \FunctionTok{dnorm}\NormalTok{(log\_mu[i], tau)}
\NormalTok{  \}}

  \DocumentationTok{\#\# quantities of interest}
    \CommentTok{\# predicted emissions}
    \DocumentationTok{\#\# from the JAGS primer: }
      \CommentTok{\# If you have two product symbols in the conditional distribution with different indices}
        \CommentTok{\# ...and two subscripts in the quantity of interest i.e. quantity[i, j] }
        \CommentTok{\# ...then this dual product is specified in JAGS using nested for loops:}
    \ControlFlowTok{for}\NormalTok{(i }\ControlFlowTok{in} \DecValTok{1}\SpecialCharTok{:}\FunctionTok{length}\NormalTok{(log.n.input.centered.pred)) \{}
      \ControlFlowTok{for}\NormalTok{(j }\ControlFlowTok{in} \DecValTok{1}\SpecialCharTok{:}\NormalTok{n.sites)\{}
\NormalTok{        log\_mu\_site\_pred[i, j] }\OtherTok{\textless{}{-}}\NormalTok{ alpha[j] }\SpecialCharTok{+}\NormalTok{ beta }\SpecialCharTok{*}\NormalTok{ log.n.input.centered.pred[i]}
\NormalTok{      \} }\CommentTok{\# end j}
\NormalTok{    \} }\CommentTok{\# end i}
\NormalTok{\}}
\end{Highlighting}
\end{Shaded}

\hypertarget{implement-jags-model-1}{%
\paragraph{Implement JAGS Model}\label{implement-jags-model-1}}

\begin{Shaded}
\begin{Highlighting}[]
\DocumentationTok{\#\#\#\#\#\#\#\#\#\#\#\#\#\#\#\#\#\#\#\#\#\#\#\#\#\#\#\#\#\#\#\#\#\#\#\#\#\#\#\#\#\#\#\#\#\#\#\#\#\#\#\#\#\#\#\#\#\#\#\#\#\#\#\#\#\#}
\CommentTok{\# insert JAGS model code into an R script}
\DocumentationTok{\#\#\#\#\#\#\#\#\#\#\#\#\#\#\#\#\#\#\#\#\#\#\#\#\#\#\#\#\#\#\#\#\#\#\#\#\#\#\#\#\#\#\#\#\#\#\#\#\#\#\#\#\#\#\#\#\#\#\#\#\#\#\#\#\#\#}
\NormalTok{\{ }\CommentTok{\# Extra bracket needed only for R markdown files {-} see answers}
  \FunctionTok{sink}\NormalTok{(}\StringTok{"NO2JAGS\_nopooled.R"}\NormalTok{) }\CommentTok{\# This is the file name for the jags code}
  \FunctionTok{cat}\NormalTok{(}\StringTok{"}
\StringTok{  model\{}
\StringTok{    \# priors}
\StringTok{    \# allow the intercept alpha to vary across sites}
\StringTok{      for(j in 1:n.sites)\{}
\StringTok{        alpha[j] \textasciitilde{} dnorm(0,1E{-}6) }
\StringTok{      \}}
\StringTok{    \# the slope beta is constant across sites}
\StringTok{    beta \textasciitilde{} dnorm(0,1E{-}6)}
\StringTok{    sigma \textasciitilde{} dunif(0,100)}
\StringTok{    tau \textless{}{-} 1/sigma\^{}2}
\StringTok{  }
\StringTok{    \# likelihood}
\StringTok{    for(i in 1:length(log.emission)) \{}
\StringTok{      log\_mu[i] \textless{}{-} alpha[group[i]] + beta * log.n.input.centered[i]}
\StringTok{      log.emission[i] \textasciitilde{} dnorm(log\_mu[i], tau)}
\StringTok{    \}}
\StringTok{  }
\StringTok{    \#\# quantities of interest}
\StringTok{      \# predicted emissions}
\StringTok{      \#\# from the JAGS primer: }
\StringTok{        \# If you have two product symbols in the conditional distribution with different indices}
\StringTok{          \# ...and two subscripts in the quantity of interest i.e. quantity[i, j] }
\StringTok{          \# ...then this dual product is specified in JAGS using nested for loops:}
\StringTok{      for(i in 1:length(log.n.input.centered.pred)) \{}
\StringTok{        for(j in 1:n.sites)\{}
\StringTok{          log\_mu\_site\_pred[i, j] \textless{}{-} alpha[j] + beta * log.n.input.centered.pred[i]}
\StringTok{        \} \# end j}
\StringTok{      \} \# end i}
\StringTok{  \}}
\StringTok{  "}\NormalTok{, }\AttributeTok{fill =} \ConstantTok{TRUE}\NormalTok{)}
  \FunctionTok{sink}\NormalTok{()}
\NormalTok{\}}
\DocumentationTok{\#\#\#\#\#\#\#\#\#\#\#\#\#\#\#\#\#\#\#\#\#\#\#\#\#\#\#\#\#\#\#\#\#\#\#\#\#\#\#\#\#\#\#\#\#\#\#\#\#\#\#\#\#\#\#\#\#\#\#\#\#\#\#\#}
\CommentTok{\# implement model}
\DocumentationTok{\#\#\#\#\#\#\#\#\#\#\#\#\#\#\#\#\#\#\#\#\#\#\#\#\#\#\#\#\#\#\#\#\#\#\#\#\#\#\#\#\#\#\#\#\#\#\#\#\#\#\#\#\#\#\#\#\#\#\#\#\#\#\#\#\#\#}
\CommentTok{\# specify 3 scalars, n.adapt, n.update, and n.iter}
\CommentTok{\# n.adapt = number of iterations that JAGS will use to choose the sampler }
  \CommentTok{\# and to assure optimum mixing of the MCMC chain}
\NormalTok{n.adapt }\OtherTok{=} \DecValTok{1000}
\CommentTok{\# n.update = number of iterations that will be discarded to allow the chain to }
\CommentTok{\#   converge before iterations are stored (aka, burn{-}in)}
\NormalTok{n.update }\OtherTok{=} \DecValTok{10000}
\CommentTok{\# n.iter = number of iterations that will be stored in the }
  \CommentTok{\# final chain as samples from the posterior distribution}
\NormalTok{n.iter }\OtherTok{=} \DecValTok{10000}
\DocumentationTok{\#\#\#\#\#\#\#\#\#\#\#\#\#\#\#\#\#\#\#\#\#\#}
\CommentTok{\# Call to JAGS}
\DocumentationTok{\#\#\#\#\#\#\#\#\#\#\#\#\#\#\#\#\#\#\#\#\#\#}
\NormalTok{jm }\OtherTok{=}\NormalTok{ rjags}\SpecialCharTok{::}\FunctionTok{jags.model}\NormalTok{(}
  \AttributeTok{file =} \StringTok{"NO2JAGS\_nopooled.R"}
\NormalTok{  , }\AttributeTok{data =}\NormalTok{ data}
\NormalTok{  , }\AttributeTok{inits =}\NormalTok{ inits}
\NormalTok{  , }\AttributeTok{n.chains =} \FunctionTok{length}\NormalTok{(inits)}
\NormalTok{  , }\AttributeTok{n.adapt =}\NormalTok{ n.adapt}
\NormalTok{)}
\end{Highlighting}
\end{Shaded}

\begin{verbatim}
## Compiling model graph
##    Resolving undeclared variables
##    Allocating nodes
## Graph information:
##    Observed stochastic nodes: 563
##    Unobserved stochastic nodes: 109
##    Total graph size: 15372
## 
## Initializing model
\end{verbatim}

\begin{Shaded}
\begin{Highlighting}[]
\NormalTok{stats}\SpecialCharTok{::}\FunctionTok{update}\NormalTok{(jm, }\AttributeTok{n.iter =}\NormalTok{ n.update)}
\CommentTok{\# save the coda object (more precisely, an mcmc.list object) to R as "zc"}
\NormalTok{zc\_nopooled }\OtherTok{=}\NormalTok{ rjags}\SpecialCharTok{::}\FunctionTok{coda.samples}\NormalTok{(}
  \AttributeTok{model =}\NormalTok{ jm}
\NormalTok{  , }\AttributeTok{variable.names =} \FunctionTok{c}\NormalTok{(}\StringTok{"alpha"}\NormalTok{, }\StringTok{"beta"}\NormalTok{, }\StringTok{"sigma"}\NormalTok{, }\StringTok{"tau"}\NormalTok{, }\StringTok{"log\_mu\_site\_pred"}\NormalTok{)}
  \CommentTok{\# , variable.names = c("a", "b", "p")}
\NormalTok{  , }\AttributeTok{n.iter =}\NormalTok{ n.iter}
\NormalTok{  , }\AttributeTok{n.thin =} \DecValTok{1}
\NormalTok{)}
\end{Highlighting}
\end{Shaded}

\hypertarget{model-output-1}{%
\paragraph{Model Output}\label{model-output-1}}

Produce trace plots of the chains for model parameters, excluding
\(\boldsymbol{\alpha}\) and a summary table of these same parameters.
Assess convergence and look at the effective sample sizes for each of
these parameters. Do you think any of the chains need to be run for
longer and if so why?

\begin{Shaded}
\begin{Highlighting}[]
\DocumentationTok{\#\#\#\#\#\#\#\#\#\#\#\#\#\#\#\#\#\#\#\#\#}
\CommentTok{\# check output}
\DocumentationTok{\#\#\#\#\#\#\#\#\#\#\#\#\#\#\#\#\#\#\#\#\#}
\CommentTok{\# trace plot}
\NormalTok{MCMCvis}\SpecialCharTok{::}\FunctionTok{MCMCtrace}\NormalTok{(zc\_nopooled, }\AttributeTok{params =} \FunctionTok{c}\NormalTok{(}\StringTok{"beta"}\NormalTok{, }\StringTok{"sigma"}\NormalTok{), }\AttributeTok{pdf =} \ConstantTok{FALSE}\NormalTok{)}
\end{Highlighting}
\end{Shaded}

\includegraphics{C:/Data/ESS575/ess575_MultiLevelRegressionLab/src/../MultiLevelRegressionLab_England_files/figure-latex/unnamed-chunk-18-1.pdf}

\begin{Shaded}
\begin{Highlighting}[]
\CommentTok{\# summary}
\NormalTok{MCMCvis}\SpecialCharTok{::}\FunctionTok{MCMCsummary}\NormalTok{(zc\_nopooled, }\AttributeTok{params =} \FunctionTok{c}\NormalTok{(}\StringTok{"alpha"}\NormalTok{, }\StringTok{"beta"}\NormalTok{, }\StringTok{"sigma"}\NormalTok{)) }\SpecialCharTok{\%\textgreater{}\%} 
  \FunctionTok{data.frame}\NormalTok{() }\SpecialCharTok{\%\textgreater{}\%} 
\NormalTok{  dplyr}\SpecialCharTok{::}\FunctionTok{slice\_tail}\NormalTok{(}\AttributeTok{n =} \DecValTok{6}\NormalTok{)}
\end{Highlighting}
\end{Shaded}

\begin{verbatim}
##                   mean         sd      X2.5.        X50.      X97.5. Rhat n.eff
## alpha[104] -0.07388139 0.39108974 -0.8336653 -0.07458111  0.69397368    1 30436
## alpha[105] -0.39360170 0.42330650 -1.2244965 -0.38952922  0.43008892    1 29278
## alpha[106] -0.65723136 0.34924852 -1.3415584 -0.65741672  0.03283677    1 27740
## alpha[107] -1.09279135 0.51707445 -2.1000701 -1.09267618 -0.08456387    1 31241
## beta        0.85088341 0.10227878  0.6513408  0.85081898  1.05147326    1  8429
## sigma       1.03124878 0.03406659  0.9668343  1.03053671  1.10032086    1 13556
\end{verbatim}

\begin{Shaded}
\begin{Highlighting}[]
\CommentTok{\# Caterpillar plots}
\NormalTok{MCMCvis}\SpecialCharTok{::}\FunctionTok{MCMCplot}\NormalTok{(zc\_nopooled, }\AttributeTok{params =} \FunctionTok{c}\NormalTok{(}\StringTok{"beta"}\NormalTok{, }\StringTok{"sigma"}\NormalTok{), }\AttributeTok{xlim =} \FunctionTok{c}\NormalTok{(}\SpecialCharTok{{-}}\FloatTok{0.5}\NormalTok{,}\FloatTok{1.5}\NormalTok{) )}
\end{Highlighting}
\end{Shaded}

\includegraphics{C:/Data/ESS575/ess575_MultiLevelRegressionLab/src/../MultiLevelRegressionLab_England_files/figure-latex/unnamed-chunk-18-2.pdf}

Assess convergence and look at the effective sample sizes for each of
these parameters. Do you think any of the chains need to be run for
longer and if so why?

\textcolor{violet}{fill this in!!!!}

Make a horizontal caterpillar plot for the the \(\boldsymbol{\alpha}\).

\begin{Shaded}
\begin{Highlighting}[]
\CommentTok{\# Caterpillar plots}
\NormalTok{MCMCvis}\SpecialCharTok{::}\FunctionTok{MCMCplot}\NormalTok{(}
\NormalTok{  zc\_nopooled}
\NormalTok{  , }\AttributeTok{params =} \FunctionTok{c}\NormalTok{(}\StringTok{"alpha"}\NormalTok{)}
\NormalTok{  , }\AttributeTok{horiz =} \ConstantTok{FALSE}
\NormalTok{  , }\AttributeTok{ylim =} \FunctionTok{c}\NormalTok{(}\SpecialCharTok{{-}}\DecValTok{6}\NormalTok{,}\DecValTok{5}\NormalTok{)}
  \CommentTok{\# Number specifying size of text for parameter labels on axis.}
\NormalTok{  , }\AttributeTok{sz\_labels =} \FloatTok{0.6}
  \CommentTok{\# Number specifying size of points represents posterior medians.}
\NormalTok{  , }\AttributeTok{sz\_med =} \FloatTok{0.7}
  \CommentTok{\# Number specifying thickness of 50 percent CI line (thicker line).}
\NormalTok{  , }\AttributeTok{sz\_thick =} \DecValTok{2}
  \CommentTok{\# Number specifying thickness of 95 percent CI line (thinner line).}
\NormalTok{  , }\AttributeTok{sz\_thin =} \DecValTok{1}
\NormalTok{)}
\end{Highlighting}
\end{Shaded}

\includegraphics{C:/Data/ESS575/ess575_MultiLevelRegressionLab/src/../MultiLevelRegressionLab_England_files/figure-latex/unnamed-chunk-20-1.pdf}

\hypertarget{question-6-1}{%
\subsubsection{Question 6}\label{question-6-1}}

How is the model able to estimate intercepts for sites where there is
only a single x value, or even sites where there is only a single
observation at all?

\textcolor{violet}{The model is able to estimate intercepts for sites where there is only a single data record because, in this model, the slope ($\beta$) is calculated using data from all sites. That is, the slope is assumed to be constant across sites. If we also allowed the slope ($\beta$) to vary across sites, then the model would require more than one data point to estimate site-specifice intercept and slope.}

\hypertarget{visualizing-the-no-pool-model-predictions}{%
\subsection{Visualizing the no-pool model
predictions}\label{visualizing-the-no-pool-model-predictions}}

We modify the \texttt{MCMCpstr} code from the previous model to produce
a data frame of the median and 95\% HDPI credible intervals of
\(\textrm{N} _2 \textrm{O}\) emission predictions for each site.
\texttt{MCMCpstr} preserves the shape of the parameter from your JAGS
model, which can be very handy in certain situations. Here,
\texttt{pred1} is a list whose first element is a 3D-array. This array's
rows are fertilizer inputs, columns are sites, and z-values are the
quantities produced by the \texttt{hdi} function, which in this case is
the lower and upper credible interval. You can \texttt{str} the
\texttt{pred1{[}{[}1{]}{]}} object to see this for yourself. For
plotting purposes though, we would like a data frame with columns for
site, fertilizer input, the posterior's median emission, and the
posterior's lower and upper HDPI credible intervals. This can be made
easily using the \texttt{melt} function to go from wide to long followed
by the \texttt{spread} function to make separate columns for the lower
and upper bounds. Then we rely on \texttt{select} and \texttt{arrange}
to order the data properly and keep the relevant columns. Lastly, we use
\texttt{cbind} to make the data frame we seek, taking advantage of the
fact that \texttt{n.input.pred} will repeat each site, which is exactly
what we want it to do.

\begin{Shaded}
\begin{Highlighting}[]
\CommentTok{\# HDI}
\NormalTok{pred1 }\OtherTok{\textless{}{-}}\NormalTok{ MCMCvis}\SpecialCharTok{::}\FunctionTok{MCMCpstr}\NormalTok{(}
\NormalTok{  zc\_nopooled}
\NormalTok{  , }\AttributeTok{params =} \StringTok{"log\_mu\_site\_pred"}
\NormalTok{  , }\AttributeTok{func =} \ControlFlowTok{function}\NormalTok{(x) HDInterval}\SpecialCharTok{::}\FunctionTok{hdi}\NormalTok{(x, .}\DecValTok{95}\NormalTok{)}
\NormalTok{)}
\CommentTok{\# median}
\NormalTok{pred2 }\OtherTok{\textless{}{-}}\NormalTok{ MCMCvis}\SpecialCharTok{::}\FunctionTok{MCMCpstr}\NormalTok{(}
\NormalTok{  zc\_nopooled}
\NormalTok{  , }\AttributeTok{params =} \StringTok{"log\_mu\_site\_pred"}
\NormalTok{  , }\AttributeTok{func =}\NormalTok{ median}
\NormalTok{)}
\CommentTok{\# create data frame}
\NormalTok{pred1.df }\OtherTok{\textless{}{-}}\NormalTok{ reshape2}\SpecialCharTok{::}\FunctionTok{melt}\NormalTok{(pred1[[}\DecValTok{1}\NormalTok{]], }\AttributeTok{as.is =} \ConstantTok{TRUE}\NormalTok{, }\AttributeTok{varnames =} \FunctionTok{c}\NormalTok{(}\StringTok{"x"}\NormalTok{, }\StringTok{"group.index"}\NormalTok{, }\StringTok{"metric"}\NormalTok{)) }\SpecialCharTok{\%\textgreater{}\%} 
  \FunctionTok{spread}\NormalTok{(metric, value) }\SpecialCharTok{\%\textgreater{}\%}
  \FunctionTok{arrange}\NormalTok{(group.index, x) }\SpecialCharTok{\%\textgreater{}\%}
\NormalTok{  dplyr}\SpecialCharTok{::}\FunctionTok{select}\NormalTok{(group.index, lower, upper)}
\NormalTok{pred2.df }\OtherTok{\textless{}{-}}\NormalTok{ reshape2}\SpecialCharTok{::}\FunctionTok{melt}\NormalTok{(pred2[[}\DecValTok{1}\NormalTok{]], }\AttributeTok{as.is =} \ConstantTok{TRUE}\NormalTok{, }\AttributeTok{varnames =} \FunctionTok{c}\NormalTok{(}\StringTok{"x"}\NormalTok{, }\StringTok{"group.index"}\NormalTok{), }\AttributeTok{value.name =} \StringTok{"median"}\NormalTok{) }\SpecialCharTok{\%\textgreater{}\%}
  \FunctionTok{arrange}\NormalTok{(group.index, x) }\SpecialCharTok{\%\textgreater{}\%} 
\NormalTok{  dplyr}\SpecialCharTok{::}\FunctionTok{select}\NormalTok{(median)}
\CommentTok{\# cbind}
\NormalTok{logpred.nopool.df }\OtherTok{\textless{}{-}} \FunctionTok{cbind}\NormalTok{(}\AttributeTok{log.n.input.pred =} \FunctionTok{log}\NormalTok{(n.input.pred), pred1.df, pred2.df)}
\end{Highlighting}
\end{Shaded}

To add the predictions to the plots for each site we use
\texttt{geom\_line} and \texttt{geom\_ribbon} again, in combination with
\texttt{facet\_wrap}.

\begin{Shaded}
\begin{Highlighting}[]
\NormalTok{g2 }\SpecialCharTok{+}
  \FunctionTok{geom\_line}\NormalTok{(}
    \AttributeTok{data =}\NormalTok{ logpred.nopool.df}
\NormalTok{    , }\AttributeTok{mapping =} \FunctionTok{aes}\NormalTok{(}\AttributeTok{x =}\NormalTok{ log.n.input.pred, }\AttributeTok{y =}\NormalTok{ median)}
\NormalTok{  ) }\SpecialCharTok{+}
  \FunctionTok{geom\_ribbon}\NormalTok{(}
    \AttributeTok{data =}\NormalTok{ logpred.nopool.df}
\NormalTok{    , }\AttributeTok{mapping =} \FunctionTok{aes}\NormalTok{(}\AttributeTok{x =}\NormalTok{ log.n.input.pred, }\AttributeTok{ymin =}\NormalTok{ lower, }\AttributeTok{ymax =}\NormalTok{ upper)}
\NormalTok{    , }\AttributeTok{alpha =} \FloatTok{0.2}
\NormalTok{    , }\AttributeTok{fill =} \StringTok{"yellow"}
\NormalTok{  ) }\SpecialCharTok{+}
  \FunctionTok{facet\_wrap}\NormalTok{(}\SpecialCharTok{\textasciitilde{}}\NormalTok{group.index)}
\end{Highlighting}
\end{Shaded}

\includegraphics{C:/Data/ESS575/ess575_MultiLevelRegressionLab/src/../MultiLevelRegressionLab_England_files/figure-latex/unnamed-chunk-23-1.pdf}

\hypertarget{random-intercepts}{%
\section{Random Intercepts}\label{random-intercepts}}

\hypertarget{diagramming-and-writing-the-random-intercepts-model}{%
\subsection{Diagramming and writing the random intercepts
model}\label{diagramming-and-writing-the-random-intercepts-model}}

So far you have either ignored site completely (the pooled model) or
treated all the site intercepts as independent from one another (the
no-pool model). Now you are going to treat the site intercepts as
partially pooled, meaning you will model them as coming from a common
distribution. In other words, you will treat these intercepts in your
model as a group level effect (aka, random effect). Hence, this model is
often called a random-intercepts model. Like in the no-pool model, the
deterministic portion of this model remains a linearized power function,
but two subscripts are required: \(i\) which indexes the measurement
within sites and \(j\) which indexes site itself. However, unlike the
no-pool model, assume that these intercepts are drawn from a
distribution with mean \(\mu_{\alpha}\) and variance
\(\varsigma_{\alpha}^2\).

\hypertarget{question-1-2}{%
\subsubsection{Question 1}\label{question-1-2}}

Draw a Bayesian network for a linear regression model of
\(\textrm{N} _2 \textrm{O}\) emission (\(y_{ij}\)) on fertilizer
addition (\(x_{ij}\)).

\begin{figure}

{\centering \includegraphics[width=0.5\linewidth,height=0.5\textheight]{../data/DAG3} 

}

\caption{DAG}\label{fig:unnamed-chunk-25}
\end{figure}

\hypertarget{question-2-2}{%
\subsubsection{Question 2}\label{question-2-2}}

Write out the posterior and joint distribution for a linear regression
model of \(\textrm{N} _2 \textrm{O}\) emission (\(y_{ij}\)) on
fertilizer addition (\(x_{ij}\)). Start by using generic
\texttt{{[}\ {]}}. Use \(\sigma^{2}\) and, \(\varsigma^{2}\) to
represent the uncertainty in your model realizing that you might need
moment matching when you choose a specific distribution.

\[
g \bigl(\alpha_{j},\beta,\log(x_{ij}) \bigr) = \alpha_{j} + \beta \bigl(\log(x_ij) \bigr)
\]

Joint:

\[
\bigl[ \alpha_{j},\beta, \mu_{\alpha}, \sigma^2, \varsigma_{\alpha}^2  \mid \boldsymbol{y} \bigr] \propto \prod_{i=1}^{n} \prod_{j=1}^{J}  \bigl[ \log(y_{ij})\mid g\bigl(\alpha_{j}, \beta,\log(x_{ij})\bigr),\sigma^{2} \bigr] \bigl[\alpha_{j} \mid \mu_{\alpha}, \varsigma_{\alpha}^2 \bigr] \bigl[ \beta \bigr] \bigl[ \sigma  \bigr] \bigl[ \varsigma  \bigr]
\]

\hypertarget{question-3-2}{%
\subsubsection{Question 3}\label{question-3-2}}

Finish by choosing specific distributions for likelihoods and priors.
You will use the math in the answer as a template to code your model in
the subsequent exercises.

\[
\begin{aligned}
\bigl[ \alpha_{j},\beta, \mu_{\alpha}, \sigma^2, \varsigma_{\alpha}^2  \mid \boldsymbol{y} \bigr] \propto \prod_{i=1}^{n}  \prod_{j=1}^{J} {\sf normal} \bigr( \log(y_{ij}) \mid g \bigl( \alpha_{j}, \beta, \log(x_{ij})  \bigr), \sigma^{2}\bigr)\\
\times \; {\sf normal} \bigr(\alpha_{j} \mid \mu_{\alpha}, \varsigma_{\alpha}^2 \bigr) \\ 
\times \; {\sf normal} \bigr(\beta \mid 0,10000\bigr) \\
\times \; {\sf uniform}\bigr(\sigma \mid 0, 100 \bigl) \\
\times \; {\sf uniform}\bigr(\varsigma \mid 0, 100 \bigl)
\end{aligned}
\] \#\#\# Fitting the random intercepts model with JAGS

Now you will implement the random-intercepts model that allows the
intercept \(\alpha_{j}\) to vary by site, where each intercept is drawn
from a common distribution. Use the \texttt{data} and initial values for
JAGS provided below to allow you to concentrate on writing JAGS code for
the model.

In addition to fitting this model, we would like you to have JAGS
predict the mean logged \(\textrm{N} _2 \textrm{O}\) emissions
\textbf{for each site} as a function of soil fertilizer input, just like
you did in the no-pool model. We also would like you to predict the mean
logged \(\textrm{N} _2 \textrm{O}\) emissions and the median unlogged
\(\textrm{N} _2 \textrm{O}\) emissions as a function of soil fertilizer
input, just like you did in the pooled model. However, these predictions
should take into account \textbf{the uncertainty associated with site}.
This is equivalent to asking you to make a prediction for a new site
whose intercept \(\alpha_{j}\) is drawn from the same distribution as
the intercepts are for the actual sites themselves. To help you out we
have provided the range of \(\textrm{N} _2 \textrm{O}\) values to
predict over as the third element in the \texttt{data} list.

\begin{Shaded}
\begin{Highlighting}[]
\NormalTok{n.input.pred }\OtherTok{\textless{}{-}} \FunctionTok{seq}\NormalTok{(}\FunctionTok{min}\NormalTok{(N2OEmission}\SpecialCharTok{$}\NormalTok{n.input), }\FunctionTok{max}\NormalTok{(N2OEmission}\SpecialCharTok{$}\NormalTok{n.input), }\DecValTok{10}\NormalTok{)}
\NormalTok{n.sites }\OtherTok{\textless{}{-}} \FunctionTok{length}\NormalTok{(}\FunctionTok{unique}\NormalTok{(N2OEmission}\SpecialCharTok{$}\NormalTok{group.index))}

\NormalTok{data }\OtherTok{=} \FunctionTok{list}\NormalTok{(}
  \AttributeTok{log.emission =} \FunctionTok{log}\NormalTok{(N2OEmission}\SpecialCharTok{$}\NormalTok{emission),}
  \AttributeTok{log.n.input.centered =} \FunctionTok{log}\NormalTok{(N2OEmission}\SpecialCharTok{$}\NormalTok{n.input) }\SpecialCharTok{{-}} \FunctionTok{mean}\NormalTok{(}\FunctionTok{log}\NormalTok{(N2OEmission}\SpecialCharTok{$}\NormalTok{n.input)),}
  \AttributeTok{log.n.input.centered.pred =} \FunctionTok{log}\NormalTok{(n.input.pred) }\SpecialCharTok{{-}} \FunctionTok{mean}\NormalTok{(}\FunctionTok{log}\NormalTok{(N2OEmission}\SpecialCharTok{$}\NormalTok{n.input)),}
  \AttributeTok{group =}\NormalTok{ N2OEmission}\SpecialCharTok{$}\NormalTok{group.index,}
  \AttributeTok{n.sites =}\NormalTok{ n.sites)}

\NormalTok{inits }\OtherTok{=} \FunctionTok{list}\NormalTok{(}
  \FunctionTok{list}\NormalTok{(}\AttributeTok{alpha =} \FunctionTok{rep}\NormalTok{(}\DecValTok{0}\NormalTok{, n.sites), }\AttributeTok{beta =}\NormalTok{ .}\DecValTok{5}\NormalTok{, }\AttributeTok{sigma =} \DecValTok{50}\NormalTok{, }\AttributeTok{mu.alpha=} \DecValTok{0}\NormalTok{, }\AttributeTok{sigma.alpha =} \DecValTok{10}\NormalTok{),}
  \FunctionTok{list}\NormalTok{(}\AttributeTok{alpha =} \FunctionTok{rep}\NormalTok{(}\DecValTok{1}\NormalTok{, n.sites), }\AttributeTok{beta =} \FloatTok{1.5}\NormalTok{, }\AttributeTok{sigma =} \DecValTok{10}\NormalTok{, }\AttributeTok{mu.alpha=} \DecValTok{2}\NormalTok{, }\AttributeTok{sigma.alpha =} \DecValTok{20}\NormalTok{),}
  \FunctionTok{list}\NormalTok{(}\AttributeTok{alpha =} \FunctionTok{rep}\NormalTok{(}\SpecialCharTok{{-}}\DecValTok{1}\NormalTok{, n.sites), }\AttributeTok{beta =}\NormalTok{ .}\DecValTok{75}\NormalTok{, }\AttributeTok{sigma =} \DecValTok{20}\NormalTok{, }\AttributeTok{mu.alpha=} \SpecialCharTok{{-}}\DecValTok{1}\NormalTok{, }\AttributeTok{sigma.alpha =} \DecValTok{12}\NormalTok{))}
\end{Highlighting}
\end{Shaded}

\hypertarget{question-5-2}{%
\subsubsection{Question 5}\label{question-5-2}}

Write the code for the model. Compile the model and execute the MCMC to
produce a coda object. Produce trace plots of the chains for model
parameters, excluding \(\boldsymbol{\alpha}\) and a summary table of
these same parameters. Assess convergence and look at the effective
sample sizes for each of these parameters. Do you think any of the
chains need to be run for longer and if so why? Make a horizontal
caterpillar plot for the the \(\boldsymbol{\alpha}\).

\hypertarget{jags-model-2}{%
\paragraph{JAGS Model}\label{jags-model-2}}

\begin{Shaded}
\begin{Highlighting}[]
\DocumentationTok{\#\# JAGS Model}
\NormalTok{model\{}
  \CommentTok{\# priors}
\NormalTok{  beta }\SpecialCharTok{\textasciitilde{}} \FunctionTok{dnorm}\NormalTok{(}\DecValTok{0}\NormalTok{,}\FloatTok{1E{-}6}\NormalTok{)}
\NormalTok{  sigma }\SpecialCharTok{\textasciitilde{}} \FunctionTok{dunif}\NormalTok{(}\DecValTok{0}\NormalTok{,}\DecValTok{100}\NormalTok{)}
\NormalTok{  tau\_y }\OtherTok{\textless{}{-}} \DecValTok{1}\SpecialCharTok{/}\NormalTok{sigma}\SpecialCharTok{\^{}}\DecValTok{2}
  \CommentTok{\# alpha priors}
\NormalTok{  mu\_alpha }\SpecialCharTok{\textasciitilde{}} \FunctionTok{dnorm}\NormalTok{(}\DecValTok{0}\NormalTok{,}\FloatTok{1E{-}6}\NormalTok{)}
\NormalTok{  varsigma }\SpecialCharTok{\textasciitilde{}} \FunctionTok{dunif}\NormalTok{(}\DecValTok{0}\NormalTok{,}\DecValTok{100}\NormalTok{)}
\NormalTok{  tau\_alpha }\OtherTok{\textless{}{-}} \DecValTok{1}\SpecialCharTok{/}\NormalTok{varsigma}\SpecialCharTok{\^{}}\DecValTok{2}
  \CommentTok{\# allow the intercept alpha to vary across sites}
    \ControlFlowTok{for}\NormalTok{(j }\ControlFlowTok{in} \DecValTok{1}\SpecialCharTok{:}\NormalTok{n.sites)\{}
\NormalTok{      alpha[j] }\SpecialCharTok{\textasciitilde{}} \FunctionTok{dnorm}\NormalTok{(mu\_alpha, tau\_alpha)}
\NormalTok{    \}}

  \CommentTok{\# likelihood}
  \ControlFlowTok{for}\NormalTok{(i }\ControlFlowTok{in} \DecValTok{1}\SpecialCharTok{:}\FunctionTok{length}\NormalTok{(log.emission)) \{}
\NormalTok{    log\_mu[i] }\OtherTok{\textless{}{-}}\NormalTok{ alpha[group[i]] }\SpecialCharTok{+}\NormalTok{ beta }\SpecialCharTok{*}\NormalTok{ log.n.input.centered[i]}
\NormalTok{    log.emission[i] }\SpecialCharTok{\textasciitilde{}} \FunctionTok{dnorm}\NormalTok{(log\_mu[i], tau\_y)}
\NormalTok{  \}}

  \DocumentationTok{\#\# quantities of interest}
    \CommentTok{\# predicted emissions FOR EACH SITE}
      \DocumentationTok{\#\# from the JAGS primer: }
        \CommentTok{\# If you have two product symbols in the conditional distribution with different indices}
          \CommentTok{\# ...and two subscripts in the quantity of interest i.e. quantity[i, j] }
          \CommentTok{\# ...then this dual product is specified in JAGS using nested for loops:}
      \ControlFlowTok{for}\NormalTok{(i }\ControlFlowTok{in} \DecValTok{1}\SpecialCharTok{:}\FunctionTok{length}\NormalTok{(log.n.input.centered.pred)) \{}
        \ControlFlowTok{for}\NormalTok{(j }\ControlFlowTok{in} \DecValTok{1}\SpecialCharTok{:}\NormalTok{n.sites)\{}
\NormalTok{          log\_mu\_site\_pred[i, j] }\OtherTok{\textless{}{-}}\NormalTok{ alpha[j] }\SpecialCharTok{+}\NormalTok{ beta }\SpecialCharTok{*}\NormalTok{ log.n.input.centered.pred[i]}
\NormalTok{        \} }\CommentTok{\# end j}
\NormalTok{      \} }\CommentTok{\# end i}
    
    \CommentTok{\# predicted emissions ACROSS SITES}
\NormalTok{      alpha\_pred }\SpecialCharTok{\textasciitilde{}} \FunctionTok{dnorm}\NormalTok{(mu\_alpha, tau\_alpha)}
      \ControlFlowTok{for}\NormalTok{(i }\ControlFlowTok{in} \DecValTok{1}\SpecialCharTok{:}\FunctionTok{length}\NormalTok{(log.n.input.centered.pred))\{}
\NormalTok{        log\_mu\_pred[i] }\OtherTok{\textless{}{-}}\NormalTok{ alpha\_pred }\SpecialCharTok{+}\NormalTok{ beta }\SpecialCharTok{*}\NormalTok{ log.n.input.centered.pred[i]}
\NormalTok{        mu\_pred[i] }\OtherTok{\textless{}{-}} \FunctionTok{exp}\NormalTok{(log\_mu\_pred[i])}
\NormalTok{      \}}
\NormalTok{\}}
\end{Highlighting}
\end{Shaded}

\hypertarget{implement-jags-model-2}{%
\paragraph{Implement JAGS Model}\label{implement-jags-model-2}}

\begin{Shaded}
\begin{Highlighting}[]
\DocumentationTok{\#\#\#\#\#\#\#\#\#\#\#\#\#\#\#\#\#\#\#\#\#\#\#\#\#\#\#\#\#\#\#\#\#\#\#\#\#\#\#\#\#\#\#\#\#\#\#\#\#\#\#\#\#\#\#\#\#\#\#\#\#\#\#\#\#\#}
\CommentTok{\# insert JAGS model code into an R script}
\DocumentationTok{\#\#\#\#\#\#\#\#\#\#\#\#\#\#\#\#\#\#\#\#\#\#\#\#\#\#\#\#\#\#\#\#\#\#\#\#\#\#\#\#\#\#\#\#\#\#\#\#\#\#\#\#\#\#\#\#\#\#\#\#\#\#\#\#\#\#}
\NormalTok{\{ }\CommentTok{\# Extra bracket needed only for R markdown files {-} see answers}
  \FunctionTok{sink}\NormalTok{(}\StringTok{"NO2JAGS\_randomintrcpts.R"}\NormalTok{) }\CommentTok{\# This is the file name for the jags code}
  \FunctionTok{cat}\NormalTok{(}\StringTok{"}
\StringTok{  model\{}
\StringTok{    \# priors}
\StringTok{    beta \textasciitilde{} dnorm(0,1E{-}6)}
\StringTok{    sigma \textasciitilde{} dunif(0,100)}
\StringTok{    tau\_y \textless{}{-} 1/sigma\^{}2}
\StringTok{    \# alpha priors}
\StringTok{    mu\_alpha \textasciitilde{} dnorm(0,1E{-}6)}
\StringTok{    varsigma \textasciitilde{} dunif(0,100)}
\StringTok{    tau\_alpha \textless{}{-} 1/varsigma\^{}2}
\StringTok{    \# allow the intercept alpha to vary across sites}
\StringTok{      for(j in 1:n.sites)\{}
\StringTok{        alpha[j] \textasciitilde{} dnorm(mu\_alpha, tau\_alpha)}
\StringTok{      \}}
\StringTok{  }
\StringTok{    \# likelihood}
\StringTok{    for(i in 1:length(log.emission)) \{}
\StringTok{      log\_mu[i] \textless{}{-} alpha[group[i]] + beta * log.n.input.centered[i]}
\StringTok{      log.emission[i] \textasciitilde{} dnorm(log\_mu[i], tau\_y)}
\StringTok{    \}}
\StringTok{  }
\StringTok{    \#\# quantities of interest}
\StringTok{      \# predicted emissions FOR EACH SITE}
\StringTok{        \#\# from the JAGS primer: }
\StringTok{          \# If you have two product symbols in the conditional distribution with different indices}
\StringTok{            \# ...and two subscripts in the quantity of interest i.e. quantity[i, j] }
\StringTok{            \# ...then this dual product is specified in JAGS using nested for loops:}
\StringTok{        for(i in 1:length(log.n.input.centered.pred)) \{}
\StringTok{          for(j in 1:n.sites)\{}
\StringTok{            log\_mu\_site\_pred[i, j] \textless{}{-} alpha[j] + beta * log.n.input.centered.pred[i]}
\StringTok{          \} \# end j}
\StringTok{        \} \# end i}
\StringTok{      }
\StringTok{      \# predicted emissions ACROSS SITES}
\StringTok{        alpha\_pred \textasciitilde{} dnorm(mu\_alpha, tau\_alpha)}
\StringTok{        for(i in 1:length(log.n.input.centered.pred))\{}
\StringTok{          log\_mu\_pred[i] \textless{}{-} alpha\_pred + beta * log.n.input.centered.pred[i]}
\StringTok{          mu\_pred[i] \textless{}{-} exp(log\_mu\_pred[i])}
\StringTok{        \}}
\StringTok{  \}}
\StringTok{  "}\NormalTok{, }\AttributeTok{fill =} \ConstantTok{TRUE}\NormalTok{)}
  \FunctionTok{sink}\NormalTok{()}
\NormalTok{\}}
\DocumentationTok{\#\#\#\#\#\#\#\#\#\#\#\#\#\#\#\#\#\#\#\#\#\#\#\#\#\#\#\#\#\#\#\#\#\#\#\#\#\#\#\#\#\#\#\#\#\#\#\#\#\#\#\#\#\#\#\#\#\#\#\#\#\#\#\#}
\CommentTok{\# implement model}
\DocumentationTok{\#\#\#\#\#\#\#\#\#\#\#\#\#\#\#\#\#\#\#\#\#\#\#\#\#\#\#\#\#\#\#\#\#\#\#\#\#\#\#\#\#\#\#\#\#\#\#\#\#\#\#\#\#\#\#\#\#\#\#\#\#\#\#\#\#\#}
\CommentTok{\# specify 3 scalars, n.adapt, n.update, and n.iter}
\CommentTok{\# n.adapt = number of iterations that JAGS will use to choose the sampler }
  \CommentTok{\# and to assure optimum mixing of the MCMC chain}
\NormalTok{n.adapt }\OtherTok{=} \DecValTok{1000}
\CommentTok{\# n.update = number of iterations that will be discarded to allow the chain to }
\CommentTok{\#   converge before iterations are stored (aka, burn{-}in)}
\NormalTok{n.update }\OtherTok{=} \DecValTok{10000}
\CommentTok{\# n.iter = number of iterations that will be stored in the }
  \CommentTok{\# final chain as samples from the posterior distribution}
\NormalTok{n.iter }\OtherTok{=} \DecValTok{10000}
\DocumentationTok{\#\#\#\#\#\#\#\#\#\#\#\#\#\#\#\#\#\#\#\#\#\#}
\CommentTok{\# Call to JAGS}
\DocumentationTok{\#\#\#\#\#\#\#\#\#\#\#\#\#\#\#\#\#\#\#\#\#\#}
\NormalTok{jm }\OtherTok{=}\NormalTok{ rjags}\SpecialCharTok{::}\FunctionTok{jags.model}\NormalTok{(}
  \AttributeTok{file =} \StringTok{"NO2JAGS\_randomintrcpts.R"}
\NormalTok{  , }\AttributeTok{data =}\NormalTok{ data}
\NormalTok{  , }\AttributeTok{inits =}\NormalTok{ inits}
\NormalTok{  , }\AttributeTok{n.chains =} \FunctionTok{length}\NormalTok{(inits)}
\NormalTok{  , }\AttributeTok{n.adapt =}\NormalTok{ n.adapt}
\NormalTok{)}
\end{Highlighting}
\end{Shaded}

\begin{verbatim}
## Compiling model graph
##    Resolving undeclared variables
##    Allocating nodes
## Graph information:
##    Observed stochastic nodes: 563
##    Unobserved stochastic nodes: 112
##    Total graph size: 15619
## 
## Initializing model
\end{verbatim}

\begin{Shaded}
\begin{Highlighting}[]
\NormalTok{stats}\SpecialCharTok{::}\FunctionTok{update}\NormalTok{(jm, }\AttributeTok{n.iter =}\NormalTok{ n.update)}
\CommentTok{\# save the coda object (more precisely, an mcmc.list object) to R as "zc"}
\NormalTok{zc\_randomintrcpts }\OtherTok{=}\NormalTok{ rjags}\SpecialCharTok{::}\FunctionTok{coda.samples}\NormalTok{(}
  \AttributeTok{model =}\NormalTok{ jm}
\NormalTok{  , }\AttributeTok{variable.names =} \FunctionTok{c}\NormalTok{(}\StringTok{"alpha"}\NormalTok{, }\StringTok{"beta"}\NormalTok{, }\StringTok{"sigma"}\NormalTok{, }\StringTok{"mu\_alpha"}\NormalTok{, }\StringTok{"varsigma"}\NormalTok{, }\StringTok{"log\_mu\_site\_pred"}\NormalTok{, }\StringTok{"log\_mu\_pred"}\NormalTok{, }\StringTok{"mu\_pred"}\NormalTok{)}
\NormalTok{  , }\AttributeTok{n.iter =}\NormalTok{ n.iter}
\NormalTok{  , }\AttributeTok{n.thin =} \DecValTok{1}
\NormalTok{)}
\end{Highlighting}
\end{Shaded}

\hypertarget{model-output-2}{%
\paragraph{Model Output}\label{model-output-2}}

Produce trace plots of the chains for model parameters, excluding
\(\boldsymbol{\alpha}\) and a summary table of these same parameters.
Assess convergence and look at the effective sample sizes for each of
these parameters. Do you think any of the chains need to be run for
longer and if so why? Make a horizontal caterpillar plot for the the
\(\boldsymbol{\alpha}\).

\begin{Shaded}
\begin{Highlighting}[]
\DocumentationTok{\#\#\#\#\#\#\#\#\#\#\#\#\#\#\#\#\#\#\#\#\#}
\CommentTok{\# check output}
\DocumentationTok{\#\#\#\#\#\#\#\#\#\#\#\#\#\#\#\#\#\#\#\#\#}
\CommentTok{\# trace plot}
\NormalTok{MCMCvis}\SpecialCharTok{::}\FunctionTok{MCMCtrace}\NormalTok{(zc\_randomintrcpts, }\AttributeTok{params =} \FunctionTok{c}\NormalTok{(}\StringTok{"beta"}\NormalTok{, }\StringTok{"sigma"}\NormalTok{, }\StringTok{"mu\_alpha"}\NormalTok{, }\StringTok{"varsigma"}\NormalTok{), }\AttributeTok{pdf =} \ConstantTok{FALSE}\NormalTok{)}
\end{Highlighting}
\end{Shaded}

\includegraphics{C:/Data/ESS575/ess575_MultiLevelRegressionLab/src/../MultiLevelRegressionLab_England_files/figure-latex/unnamed-chunk-29-1.pdf}
\includegraphics{C:/Data/ESS575/ess575_MultiLevelRegressionLab/src/../MultiLevelRegressionLab_England_files/figure-latex/unnamed-chunk-29-2.pdf}

\begin{Shaded}
\begin{Highlighting}[]
\CommentTok{\# summary}
\NormalTok{MCMCvis}\SpecialCharTok{::}\FunctionTok{MCMCsummary}\NormalTok{(zc\_randomintrcpts, }\AttributeTok{params =} \FunctionTok{c}\NormalTok{(}\StringTok{"beta"}\NormalTok{, }\StringTok{"sigma"}\NormalTok{, }\StringTok{"mu\_alpha"}\NormalTok{, }\StringTok{"varsigma"}\NormalTok{))}
\end{Highlighting}
\end{Shaded}

\begin{verbatim}
##               mean         sd       2.5%       50%    97.5% Rhat n.eff
## beta     0.8751152 0.09381995  0.6929538 0.8741896 1.061220    1 10458
## sigma    1.0270831 0.03378976  0.9638491 1.0262135 1.096306    1 14310
## mu_alpha 0.1818220 0.13055134 -0.0726559 0.1812325 0.438162    1 20237
## varsigma 1.2096691 0.09824695  1.0335825 1.2048714 1.417870    1 11341
\end{verbatim}

\begin{Shaded}
\begin{Highlighting}[]
\CommentTok{\# Caterpillar plots}
\NormalTok{MCMCvis}\SpecialCharTok{::}\FunctionTok{MCMCplot}\NormalTok{(zc\_randomintrcpts, }\AttributeTok{params =} \FunctionTok{c}\NormalTok{(}\StringTok{"beta"}\NormalTok{, }\StringTok{"sigma"}\NormalTok{, }\StringTok{"mu\_alpha"}\NormalTok{, }\StringTok{"varsigma"}\NormalTok{), }\AttributeTok{xlim =} \FunctionTok{c}\NormalTok{(}\SpecialCharTok{{-}}\FloatTok{0.5}\NormalTok{,}\FloatTok{1.5}\NormalTok{) )}
\end{Highlighting}
\end{Shaded}

\includegraphics{C:/Data/ESS575/ess575_MultiLevelRegressionLab/src/../MultiLevelRegressionLab_England_files/figure-latex/unnamed-chunk-29-3.pdf}

Assess convergence and look at the effective sample sizes for each of
these parameters. Do you think any of the chains need to be run for
longer and if so why?

\textcolor{violet}{fill this in!!!!}

Make a horizontal caterpillar plot for the the \(\boldsymbol{\alpha}\).

\begin{Shaded}
\begin{Highlighting}[]
\CommentTok{\# Caterpillar plots}
\NormalTok{MCMCvis}\SpecialCharTok{::}\FunctionTok{MCMCplot}\NormalTok{(}
\NormalTok{  zc\_randomintrcpts}
\NormalTok{  , }\AttributeTok{params =} \FunctionTok{c}\NormalTok{(}\StringTok{"alpha"}\NormalTok{)}
\NormalTok{  , }\AttributeTok{horiz =} \ConstantTok{FALSE}
\NormalTok{  , }\AttributeTok{ylim =} \FunctionTok{c}\NormalTok{(}\SpecialCharTok{{-}}\DecValTok{6}\NormalTok{,}\DecValTok{5}\NormalTok{)}
  \CommentTok{\# Number specifying size of text for parameter labels on axis.}
\NormalTok{  , }\AttributeTok{sz\_labels =} \FloatTok{0.6}
  \CommentTok{\# Number specifying size of points represents posterior medians.}
\NormalTok{  , }\AttributeTok{sz\_med =} \FloatTok{0.7}
  \CommentTok{\# Number specifying thickness of 50 percent CI line (thicker line).}
\NormalTok{  , }\AttributeTok{sz\_thick =} \DecValTok{2}
  \CommentTok{\# Number specifying thickness of 95 percent CI line (thinner line).}
\NormalTok{  , }\AttributeTok{sz\_thin =} \DecValTok{1}
\NormalTok{)}
\end{Highlighting}
\end{Shaded}

\includegraphics{C:/Data/ESS575/ess575_MultiLevelRegressionLab/src/../MultiLevelRegressionLab_England_files/figure-latex/unnamed-chunk-30-1.pdf}

\hypertarget{visualizing-the-random-intercepts-model-predictions}{%
\subsection{Visualizing the random intercepts model
predictions}\label{visualizing-the-random-intercepts-model-predictions}}

\hypertarget{question-6-2}{%
\subsubsection{Question 6}\label{question-6-2}}

Modify code from the pooled and no-pool models to visualize the model
predictions. For the site-level predictions, add a dotted line showing
the posterior median of \(\textrm{N} _2 \textrm{O}\) emission from the
no-pool model. What do you notice about the uncertainty of predicted log
emissions from the random intercept model you fit here compared to the
predictions of the pooled model you did in problem 1? Explain the
difference.

\hypertarget{pooled-1}{%
\paragraph{Pooled}\label{pooled-1}}

\begin{Shaded}
\begin{Highlighting}[]
\CommentTok{\# HDI}
\NormalTok{pred1 }\OtherTok{\textless{}{-}}\NormalTok{ MCMCvis}\SpecialCharTok{::}\FunctionTok{MCMCpstr}\NormalTok{(}
\NormalTok{  zc\_randomintrcpts}
\NormalTok{  , }\AttributeTok{params =} \FunctionTok{c}\NormalTok{(}\StringTok{"log\_mu\_pred"}\NormalTok{, }\StringTok{"mu\_pred"}\NormalTok{)}
\NormalTok{  , }\AttributeTok{func =} \ControlFlowTok{function}\NormalTok{(x) HDInterval}\SpecialCharTok{::}\FunctionTok{hdi}\NormalTok{(x, .}\DecValTok{95}\NormalTok{)}
\NormalTok{)}
\CommentTok{\# median}
\NormalTok{pred2 }\OtherTok{\textless{}{-}}\NormalTok{ MCMCvis}\SpecialCharTok{::}\FunctionTok{MCMCpstr}\NormalTok{(}
\NormalTok{  zc\_randomintrcpts}
\NormalTok{  , }\AttributeTok{params =} \FunctionTok{c}\NormalTok{(}\StringTok{"log\_mu\_pred"}\NormalTok{, }\StringTok{"mu\_pred"}\NormalTok{)}
\NormalTok{  , }\AttributeTok{func =}\NormalTok{ median}
\NormalTok{)}
\CommentTok{\# put in data frame}
\NormalTok{pred.randomintrcpts.df }\OtherTok{\textless{}{-}}\NormalTok{ dplyr}\SpecialCharTok{::}\FunctionTok{bind\_cols}\NormalTok{(}
\NormalTok{  n.input.pred}
\NormalTok{  , }\FunctionTok{data.frame}\NormalTok{(pred1}\SpecialCharTok{$}\NormalTok{mu\_pred)}
\NormalTok{  , }\AttributeTok{median =}\NormalTok{ pred2}\SpecialCharTok{$}\NormalTok{mu\_pred}
\NormalTok{)}
\NormalTok{lpred.randomintrcpts.df }\OtherTok{\textless{}{-}}\NormalTok{ dplyr}\SpecialCharTok{::}\FunctionTok{bind\_cols}\NormalTok{(}
  \AttributeTok{log.n.input.pred =} \FunctionTok{log}\NormalTok{(n.input.pred)}
\NormalTok{  , }\FunctionTok{data.frame}\NormalTok{(pred1}\SpecialCharTok{$}\NormalTok{log\_mu\_pred)}
\NormalTok{  , }\AttributeTok{median =}\NormalTok{ pred2}\SpecialCharTok{$}\NormalTok{log\_mu\_pred}
\NormalTok{)}
\end{Highlighting}
\end{Shaded}

Plot the predictions

\begin{Shaded}
\begin{Highlighting}[]
\NormalTok{g5 }\OtherTok{\textless{}{-}}\NormalTok{ g1 }\SpecialCharTok{+}
  \FunctionTok{geom\_line}\NormalTok{(}
    \AttributeTok{data =}\NormalTok{ pred.randomintrcpts.df}
\NormalTok{    , }\AttributeTok{mapping =} \FunctionTok{aes}\NormalTok{(}\AttributeTok{x =}\NormalTok{ n.input.pred, }\AttributeTok{y =}\NormalTok{ median)}
\NormalTok{  ) }\SpecialCharTok{+}
  \FunctionTok{geom\_ribbon}\NormalTok{(}
    \AttributeTok{data =}\NormalTok{ pred.randomintrcpts.df}
\NormalTok{    , }\AttributeTok{mapping =} \FunctionTok{aes}\NormalTok{(}\AttributeTok{x =}\NormalTok{ n.input.pred, }\AttributeTok{ymin =}\NormalTok{ lower, }\AttributeTok{ymax =}\NormalTok{ upper)}
\NormalTok{    , }\AttributeTok{alpha =} \FloatTok{0.2}
\NormalTok{    , }\AttributeTok{fill =} \StringTok{"yellow"}
\NormalTok{  )}

\NormalTok{g6 }\OtherTok{\textless{}{-}}\NormalTok{ g2 }\SpecialCharTok{+}
  \FunctionTok{geom\_line}\NormalTok{(}
    \AttributeTok{data =}\NormalTok{ lpred.randomintrcpts.df}
\NormalTok{    , }\AttributeTok{mapping =} \FunctionTok{aes}\NormalTok{(}\AttributeTok{x =}\NormalTok{ log.n.input.pred, }\AttributeTok{y =}\NormalTok{ median)}
\NormalTok{  ) }\SpecialCharTok{+}
  \FunctionTok{geom\_ribbon}\NormalTok{(}
    \AttributeTok{data =}\NormalTok{ lpred.randomintrcpts.df}
\NormalTok{    , }\AttributeTok{mapping =} \FunctionTok{aes}\NormalTok{(}\AttributeTok{x =}\NormalTok{ log.n.input.pred, }\AttributeTok{ymin =}\NormalTok{ lower, }\AttributeTok{ymax =}\NormalTok{ upper)}
\NormalTok{    , }\AttributeTok{alpha =} \FloatTok{0.2}
\NormalTok{    , }\AttributeTok{fill =} \StringTok{"yellow"}
\NormalTok{  )}

\NormalTok{gridExtra}\SpecialCharTok{::}\FunctionTok{grid.arrange}\NormalTok{(g5, g6, }\AttributeTok{nrow =} \DecValTok{1}\NormalTok{)}
\end{Highlighting}
\end{Shaded}

\includegraphics{C:/Data/ESS575/ess575_MultiLevelRegressionLab/src/../MultiLevelRegressionLab_England_files/figure-latex/unnamed-chunk-33-1.pdf}

What do you notice about the uncertainty of predicted log emissions from
the random intercept model you fit here compared to the predictions of
the pooled model you did in problem 1? Explain the difference.

\textcolor{violet}{fill this in!!!!}

\hypertarget{non-pooled-1}{%
\paragraph{Non-Pooled}\label{non-pooled-1}}

Plot the predictions

\begin{Shaded}
\begin{Highlighting}[]
\CommentTok{\# HDI}
\NormalTok{pred1 }\OtherTok{\textless{}{-}}\NormalTok{ MCMCvis}\SpecialCharTok{::}\FunctionTok{MCMCpstr}\NormalTok{(}
\NormalTok{  zc\_randomintrcpts}
\NormalTok{  , }\AttributeTok{params =} \StringTok{"log\_mu\_site\_pred"}
\NormalTok{  , }\AttributeTok{func =} \ControlFlowTok{function}\NormalTok{(x) HDInterval}\SpecialCharTok{::}\FunctionTok{hdi}\NormalTok{(x, .}\DecValTok{95}\NormalTok{)}
\NormalTok{)}
\CommentTok{\# median}
\NormalTok{pred2 }\OtherTok{\textless{}{-}}\NormalTok{ MCMCvis}\SpecialCharTok{::}\FunctionTok{MCMCpstr}\NormalTok{(}
\NormalTok{  zc\_randomintrcpts}
\NormalTok{  , }\AttributeTok{params =} \StringTok{"log\_mu\_site\_pred"}
\NormalTok{  , }\AttributeTok{func =}\NormalTok{ median}
\NormalTok{)}
\CommentTok{\# create data frame}
\NormalTok{pred1.df }\OtherTok{\textless{}{-}}\NormalTok{ reshape2}\SpecialCharTok{::}\FunctionTok{melt}\NormalTok{(pred1[[}\DecValTok{1}\NormalTok{]], }\AttributeTok{as.is =} \ConstantTok{TRUE}\NormalTok{, }\AttributeTok{varnames =} \FunctionTok{c}\NormalTok{(}\StringTok{"x"}\NormalTok{, }\StringTok{"group.index"}\NormalTok{, }\StringTok{"metric"}\NormalTok{)) }\SpecialCharTok{\%\textgreater{}\%} 
  \FunctionTok{spread}\NormalTok{(metric, value) }\SpecialCharTok{\%\textgreater{}\%}
  \FunctionTok{arrange}\NormalTok{(group.index, x) }\SpecialCharTok{\%\textgreater{}\%}
\NormalTok{  dplyr}\SpecialCharTok{::}\FunctionTok{select}\NormalTok{(group.index, lower, upper)}
\NormalTok{pred2.df }\OtherTok{\textless{}{-}}\NormalTok{ reshape2}\SpecialCharTok{::}\FunctionTok{melt}\NormalTok{(pred2[[}\DecValTok{1}\NormalTok{]], }\AttributeTok{as.is =} \ConstantTok{TRUE}\NormalTok{, }\AttributeTok{varnames =} \FunctionTok{c}\NormalTok{(}\StringTok{"x"}\NormalTok{, }\StringTok{"group.index"}\NormalTok{), }\AttributeTok{value.name =} \StringTok{"median"}\NormalTok{) }\SpecialCharTok{\%\textgreater{}\%}
  \FunctionTok{arrange}\NormalTok{(group.index, x) }\SpecialCharTok{\%\textgreater{}\%} 
\NormalTok{  dplyr}\SpecialCharTok{::}\FunctionTok{select}\NormalTok{(median)}
\CommentTok{\# cbind}
\NormalTok{logpred.randomintrcpts.df }\OtherTok{\textless{}{-}} \FunctionTok{cbind}\NormalTok{(}\AttributeTok{log.n.input.pred =} \FunctionTok{log}\NormalTok{(n.input.pred), pred1.df, pred2.df)}
\end{Highlighting}
\end{Shaded}

For the site-level predictions, add a dotted line showing the posterior
median of \(\textrm{N} _2 \textrm{O}\) emission from the no-pool model.

\begin{Shaded}
\begin{Highlighting}[]
\NormalTok{g2 }\SpecialCharTok{+}
  \FunctionTok{geom\_ribbon}\NormalTok{(}
    \AttributeTok{data =}\NormalTok{ logpred.randomintrcpts.df}
\NormalTok{    , }\AttributeTok{mapping =} \FunctionTok{aes}\NormalTok{(}\AttributeTok{x =}\NormalTok{ log.n.input.pred, }\AttributeTok{ymin =}\NormalTok{ lower, }\AttributeTok{ymax =}\NormalTok{ upper)}
\NormalTok{    , }\AttributeTok{alpha =} \FloatTok{0.2}
\NormalTok{    , }\AttributeTok{fill =} \StringTok{"yellow"}
\NormalTok{  ) }\SpecialCharTok{+}
  \CommentTok{\# add a dotted line showing...from the no{-}pool model}
  \FunctionTok{geom\_line}\NormalTok{(}
    \AttributeTok{data =}\NormalTok{ logpred.nopool.df}
\NormalTok{    , }\AttributeTok{mapping =} \FunctionTok{aes}\NormalTok{(}\AttributeTok{x =}\NormalTok{ log.n.input.pred, }\AttributeTok{y =}\NormalTok{ median)}
\NormalTok{    , }\AttributeTok{linetype =} \StringTok{"longdash"} \CommentTok{\# cant see dotted so changing }
\NormalTok{    , }\AttributeTok{color =} \StringTok{"midnightblue"}
\NormalTok{  ) }\SpecialCharTok{+}
  \CommentTok{\# random intercepts}
  \FunctionTok{geom\_line}\NormalTok{(}
    \AttributeTok{data =}\NormalTok{ logpred.randomintrcpts.df}
\NormalTok{    , }\AttributeTok{mapping =} \FunctionTok{aes}\NormalTok{(}\AttributeTok{x =}\NormalTok{ log.n.input.pred, }\AttributeTok{y =}\NormalTok{ median)}
\NormalTok{    , }\AttributeTok{linetype =} \StringTok{"solid"}
\NormalTok{    , }\AttributeTok{color =} \StringTok{"black"}
\NormalTok{  ) }\SpecialCharTok{+}
  \FunctionTok{facet\_wrap}\NormalTok{(}\SpecialCharTok{\textasciitilde{}}\NormalTok{group.index)}
\end{Highlighting}
\end{Shaded}

\includegraphics{C:/Data/ESS575/ess575_MultiLevelRegressionLab/src/../MultiLevelRegressionLab_England_files/figure-latex/unnamed-chunk-36-1.pdf}

\hypertarget{question-7}{%
\subsubsection{Question 7}\label{question-7}}

Why do the intercepts differ for some sites between the no-pool model
and the random-intercepts model? Is this behavior consistent? Look
closely at sites 51 and 56.

Look closely at sites 51 and 56.

\begin{Shaded}
\begin{Highlighting}[]
\NormalTok{temp\_grps }\OtherTok{\textless{}{-}} \FunctionTok{c}\NormalTok{(}\DecValTok{51}\NormalTok{, }\DecValTok{56}\NormalTok{)}
\FunctionTok{ggplot}\NormalTok{(}\AttributeTok{data =}\NormalTok{ BayesNSF}\SpecialCharTok{::}\NormalTok{N2OEmission }\SpecialCharTok{\%\textgreater{}\%}\NormalTok{ dplyr}\SpecialCharTok{::}\FunctionTok{filter}\NormalTok{(group.index }\SpecialCharTok{\%in\%}\NormalTok{ temp\_grps)) }\SpecialCharTok{+}
  \FunctionTok{geom\_point}\NormalTok{(}
    \AttributeTok{mapping =} \FunctionTok{aes}\NormalTok{(}\AttributeTok{y =} \FunctionTok{log}\NormalTok{(emission), }\AttributeTok{x =} \FunctionTok{log}\NormalTok{(n.input))}
\NormalTok{    , }\AttributeTok{alpha =} \DecValTok{3}\SpecialCharTok{/}\DecValTok{10}
\NormalTok{    , }\AttributeTok{shape =} \DecValTok{21}
\NormalTok{    , }\AttributeTok{colour =} \StringTok{"black"}
\NormalTok{    , }\AttributeTok{fill =} \StringTok{"brown"}
\NormalTok{    , }\AttributeTok{size =} \DecValTok{3}
\NormalTok{  ) }\SpecialCharTok{+}
  \FunctionTok{geom\_ribbon}\NormalTok{(}
    \AttributeTok{data =}\NormalTok{ logpred.randomintrcpts.df }\SpecialCharTok{\%\textgreater{}\%}\NormalTok{ dplyr}\SpecialCharTok{::}\FunctionTok{filter}\NormalTok{(group.index }\SpecialCharTok{\%in\%}\NormalTok{ temp\_grps)}
\NormalTok{    , }\AttributeTok{mapping =} \FunctionTok{aes}\NormalTok{(}\AttributeTok{x =}\NormalTok{ log.n.input.pred, }\AttributeTok{ymin =}\NormalTok{ lower, }\AttributeTok{ymax =}\NormalTok{ upper)}
\NormalTok{    , }\AttributeTok{alpha =} \FloatTok{0.2}
\NormalTok{    , }\AttributeTok{fill =} \StringTok{"yellow"}
\NormalTok{  ) }\SpecialCharTok{+}
  \CommentTok{\# add a dotted line showing...from the no{-}pool model}
  \FunctionTok{geom\_line}\NormalTok{(}
    \AttributeTok{data =}\NormalTok{ logpred.nopool.df }\SpecialCharTok{\%\textgreater{}\%}\NormalTok{ dplyr}\SpecialCharTok{::}\FunctionTok{filter}\NormalTok{(group.index }\SpecialCharTok{\%in\%}\NormalTok{ temp\_grps)}
\NormalTok{    , }\AttributeTok{mapping =} \FunctionTok{aes}\NormalTok{(}\AttributeTok{x =}\NormalTok{ log.n.input.pred, }\AttributeTok{y =}\NormalTok{ median)}
\NormalTok{    , }\AttributeTok{linetype =} \StringTok{"longdash"} \CommentTok{\# cant see dotted so changing }
\NormalTok{    , }\AttributeTok{color =} \StringTok{"midnightblue"}
\NormalTok{  ) }\SpecialCharTok{+}
  \CommentTok{\# random intercepts}
  \FunctionTok{geom\_line}\NormalTok{(}
    \AttributeTok{data =}\NormalTok{ logpred.randomintrcpts.df }\SpecialCharTok{\%\textgreater{}\%}\NormalTok{ dplyr}\SpecialCharTok{::}\FunctionTok{filter}\NormalTok{(group.index }\SpecialCharTok{\%in\%}\NormalTok{ temp\_grps)}
\NormalTok{    , }\AttributeTok{mapping =} \FunctionTok{aes}\NormalTok{(}\AttributeTok{x =}\NormalTok{ log.n.input.pred, }\AttributeTok{y =}\NormalTok{ median)}
\NormalTok{    , }\AttributeTok{linetype =} \StringTok{"solid"}
\NormalTok{    , }\AttributeTok{color =} \StringTok{"black"}
\NormalTok{  ) }\SpecialCharTok{+}
  \FunctionTok{facet\_wrap}\NormalTok{(}\SpecialCharTok{\textasciitilde{}}\NormalTok{group.index}
\NormalTok{    , }\AttributeTok{labeller =} \FunctionTok{labeller}\NormalTok{(}
        \AttributeTok{group.index =} \ControlFlowTok{function}\NormalTok{(x)\{}\FunctionTok{paste0}\NormalTok{(}\StringTok{"site \#"}\NormalTok{, x)\}}
\NormalTok{      )}
\NormalTok{  ) }\SpecialCharTok{+}
  \FunctionTok{labs}\NormalTok{(}
    \AttributeTok{subtitle =} \StringTok{"no{-}pool = dashed, random{-}intercepts = solid"}
\NormalTok{  ) }\SpecialCharTok{+}
  \FunctionTok{theme\_minimal}\NormalTok{() }\SpecialCharTok{+}
  \FunctionTok{theme}\NormalTok{(}
    \AttributeTok{plot.subtitle =} \FunctionTok{element\_text}\NormalTok{(}\AttributeTok{face=}\StringTok{"italic"}\NormalTok{)}
\NormalTok{  )}
\end{Highlighting}
\end{Shaded}

\includegraphics{C:/Data/ESS575/ess575_MultiLevelRegressionLab/src/../MultiLevelRegressionLab_England_files/figure-latex/unnamed-chunk-37-1.pdf}

Why do the intercepts differ for some sites between the no-pool model
and the random-intercepts model? Is this behavior consistent?

\textcolor{violet}{fill this in!!!! borrowing strength, # data points between sites}

\hypertarget{diagramming-and-writing-the-random-intercepts-group-level-effect-model}{%
\subsection{Diagramming and writing the random intercepts, group-level
effect
model}\label{diagramming-and-writing-the-random-intercepts-group-level-effect-model}}

In the previous example, we assumed that the variation in the intercept
was attributable to spatial variation among sites. We did not try to
explain that variation, we simply acknowledged that it exists. Now we
are going to ``model a parameter'' using soil carbon content data at the
site-level to explain variation in the intercepts among sites. Modify
the previous model to represent the effect of soil carbon on the
intercept using the deterministic model below to predict \(\alpha_j\).
Here, we logit transform the carbon data to ``spread them out'' mapping
0-1 to all real numbers.

\[
g_2 \bigl(\kappa,\eta, \text{logit}(w_j) \bigr) = \kappa + \eta \bigl(\text{logit}(w_ij) \bigr)
\]

\hypertarget{question-8}{%
\subsubsection{Question 8}\label{question-8}}

Draw a Bayesian network for a linear regression model of
\(\textrm{N} _2 \textrm{O}\) emission (\(y_{ij}\)) on fertilizer
addition (\(x_{ij}\)) and soil carbon content (\(w_{j}\)).

\begin{figure}

{\centering \includegraphics[width=0.5\linewidth,height=0.5\textheight]{../data/DAG4} 

}

\caption{DAG}\label{fig:unnamed-chunk-39}
\end{figure}

\hypertarget{question-9}{%
\subsubsection{Question 9}\label{question-9}}

Write out the posterior and joint distribution for a linear regression
model of \(\textrm{N} _2 \textrm{O}\) emission (\(y_{ij}\)) on
fertilizer addition (\(x_{ij}\)) and soil carbon content (\(w_{j}\)).
Choose appropriate distributions for each random variable.

\[
\begin{aligned}
g \bigl(\alpha_{j},\beta,\log(x_{ij}) \bigr) = \alpha_{j} + \beta \bigl(\log(x_ij) \bigr)\\
g_2 \bigl(\kappa,\eta, \text{logit}(w_j) \bigr) = \kappa + \eta \bigl(\text{logit}(w_ij) \bigr)
\end{aligned}
\]

Joint:

\$\$

\begin{aligned}

\bigl[ \alpha_{j},\beta, \mu_{\alpha}, \sigma^2, \varsigma_{\alpha}^2  \mid \boldsymbol{y} \bigr] \propto \prod_{i=1}^{n}  \prod_{j=1}^{J} {\sf normal} \bigr( \log(y_{ij}) \mid g \bigl( \alpha_{j}, \beta, \log(x_{ij})  \bigr), \sigma^{2}\bigr)\\
\times \; {\sf normal} \bigr(\alpha_{j} \mid g_2 \bigl(\kappa,\eta, \text{logit}(w_j) \bigr), \varsigma_{\alpha}^2 \bigr) \\ 
\times \; {\sf normal} \bigr(\beta \mid 0,10000\bigr) \\
\times \; {\sf normal} \bigr(\kappa \mid 0,10000\bigr) \\
\times \; {\sf normal} \bigr(\eta \mid 0,10000\bigr) \\
\times \; {\sf uniform}\bigr(\sigma \mid 0, 100 \bigl) \\
\times \; {\sf uniform}\bigr(\varsigma \mid 0, 100 \bigl)
\end{aligned}

\$\$

\hypertarget{fitting-the-random-intercepts-group-level-effect-model-with-jags}{%
\subsection{Fitting the random intercepts, group-level effect model with
JAGS}\label{fitting-the-random-intercepts-group-level-effect-model-with-jags}}

Modify your random intercepts model to implement the model that include
soil carbon content as covariate at the site level. Make predictions for
how mean logged \(\textrm{N} _2 \textrm{O}\) emission and median
\(\textrm{N} _2 \textrm{O}\) emission varies with respect to soil
fertilizer input \emph{the uncertainty associated with site} as you did
in the previous example. Use the \texttt{data} and initial values for
JAGS provided below to allow you to concentrate on writing JAGS code for
the model.

\begin{Shaded}
\begin{Highlighting}[]
\NormalTok{n.input.pred }\OtherTok{\textless{}{-}} \FunctionTok{seq}\NormalTok{(}
  \FunctionTok{min}\NormalTok{(BayesNSF}\SpecialCharTok{::}\NormalTok{N2OEmission}\SpecialCharTok{$}\NormalTok{n.input)}
\NormalTok{  , }\FunctionTok{max}\NormalTok{(BayesNSF}\SpecialCharTok{::}\NormalTok{N2OEmission}\SpecialCharTok{$}\NormalTok{n.input)}
\NormalTok{  , }\DecValTok{10}
\NormalTok{)}
\NormalTok{n.sites }\OtherTok{\textless{}{-}} \FunctionTok{length}\NormalTok{(}\FunctionTok{unique}\NormalTok{(BayesNSF}\SpecialCharTok{::}\NormalTok{N2OEmission}\SpecialCharTok{$}\NormalTok{group.index))}

\NormalTok{data }\OtherTok{=} \FunctionTok{list}\NormalTok{(}
  \AttributeTok{log.emission =} \FunctionTok{log}\NormalTok{(BayesNSF}\SpecialCharTok{::}\NormalTok{N2OEmission}\SpecialCharTok{$}\NormalTok{emission)}
\NormalTok{  , }\AttributeTok{log.n.input.centered =} \FunctionTok{log}\NormalTok{(BayesNSF}\SpecialCharTok{::}\NormalTok{N2OEmission}\SpecialCharTok{$}\NormalTok{n.input) }\SpecialCharTok{{-}} \FunctionTok{mean}\NormalTok{(}\FunctionTok{log}\NormalTok{(BayesNSF}\SpecialCharTok{::}\NormalTok{N2OEmission}\SpecialCharTok{$}\NormalTok{n.input))}
\NormalTok{  , }\AttributeTok{log.n.input.centered.pred =} \FunctionTok{log}\NormalTok{(n.input.pred) }\SpecialCharTok{{-}} \FunctionTok{mean}\NormalTok{(}\FunctionTok{log}\NormalTok{(BayesNSF}\SpecialCharTok{::}\NormalTok{N2OEmission}\SpecialCharTok{$}\NormalTok{n.input))}
    \CommentTok{\#divide by 100 to make data a proportion, take logit, and center}
\NormalTok{  , }\AttributeTok{w =}\NormalTok{ boot}\SpecialCharTok{::}\FunctionTok{logit}\NormalTok{(SiteCarbon}\SpecialCharTok{$}\NormalTok{mean}\SpecialCharTok{/}\DecValTok{100}\NormalTok{) }\SpecialCharTok{{-}} \FunctionTok{mean}\NormalTok{(boot}\SpecialCharTok{::}\FunctionTok{logit}\NormalTok{(SiteCarbon}\SpecialCharTok{$}\NormalTok{mean}\SpecialCharTok{/}\DecValTok{100}\NormalTok{))}
\NormalTok{  , }\AttributeTok{group =}\NormalTok{ BayesNSF}\SpecialCharTok{::}\NormalTok{N2OEmission}\SpecialCharTok{$}\NormalTok{group.index}
\NormalTok{  , }\AttributeTok{n.sites =}\NormalTok{ n.sites}
\NormalTok{)}

\NormalTok{inits }\OtherTok{=} \FunctionTok{list}\NormalTok{(}
  \FunctionTok{list}\NormalTok{(}\AttributeTok{alpha =} \FunctionTok{rep}\NormalTok{(}\DecValTok{0}\NormalTok{, n.sites), }\AttributeTok{beta =}\NormalTok{ .}\DecValTok{5}\NormalTok{, }\AttributeTok{sigma =} \DecValTok{50}\NormalTok{, }\AttributeTok{sigma.alpha =} \DecValTok{10}\NormalTok{, }\AttributeTok{eta =}\NormalTok{ .}\DecValTok{2}\NormalTok{, }\AttributeTok{kappa =}\NormalTok{ .}\DecValTok{5}\NormalTok{)}
\NormalTok{  , }\FunctionTok{list}\NormalTok{(}\AttributeTok{alpha =} \FunctionTok{rep}\NormalTok{(}\DecValTok{1}\NormalTok{, n.sites), }\AttributeTok{beta =} \FloatTok{1.5}\NormalTok{, }\AttributeTok{sigma =} \DecValTok{10}\NormalTok{, }\AttributeTok{sigma.alpha =} \DecValTok{20}\NormalTok{, }\AttributeTok{eta =} \DecValTok{3}\NormalTok{, }\AttributeTok{kappa =}\NormalTok{ .}\DecValTok{7}\NormalTok{)}
\NormalTok{  , }\FunctionTok{list}\NormalTok{(}\AttributeTok{alpha =} \FunctionTok{rep}\NormalTok{(}\SpecialCharTok{{-}}\DecValTok{1}\NormalTok{, n.sites), }\AttributeTok{beta =}\NormalTok{ .}\DecValTok{75}\NormalTok{, }\AttributeTok{sigma =} \DecValTok{20}\NormalTok{, }\AttributeTok{sigma.alpha =} \DecValTok{12}\NormalTok{, }\AttributeTok{eta =}\NormalTok{ .}\DecValTok{1}\NormalTok{, }\AttributeTok{kappa =}\NormalTok{ .}\DecValTok{3}\NormalTok{)}
\NormalTok{)}
\end{Highlighting}
\end{Shaded}

\hypertarget{question-10}{%
\subsubsection{Question 10}\label{question-10}}

Write the code for the model. Compile the model and execute the MCMC to
produce a coda object. Produce trace plots of the chains for model
parameters, excluding \(\boldsymbol{\alpha}\) and a summary table of
these same parameters. Assess convergence and look at the effective
sample sizes for each of these parameters. Do you think any of the
chains need to be run for longer and if so why? Make a horizontal
caterpillar plot for the the \(\boldsymbol{\alpha}\).

\begin{Shaded}
\begin{Highlighting}[]
\CommentTok{\# in progress}
\end{Highlighting}
\end{Shaded}


\end{document}
